\chapter{موسیقی}
% \afterpage{\newpage}
\section{تاثیر‌ آن‌ در‌ زندگی}

موسیقی درتمامی بخش‌های‌زندگی ماوجود دارد.اعم از زندگی روزمره،  کار، حتی هنگام دیدن یک فیلم و یا خواب می‌توان تاثیر آن را مشاهده کرد. بر اساس مشاهدات گوش دادن به موسیقی‌های متفاوت بر روی کاهش درد و  یا عمل کرد بهتر مغز انسان تاثیر دارد. 
در آزمایشی که بر روی کارکنان شرکت‌های مختلف کانادایی انجام شده است، مطالعات نشان می‌دهد که به طور میانگین عملکرد آن‌ها 15 درصد افزایش داشته است. علاوه بر این‌ها گوش دادن به موسیقی بر غذا خوردن هم تاثیر دارد. مطالعات نشان می‌دهد که گوش دادن به موسیقی سریع هنگام عذا خوردن، روند غذا خورد را سریع‌تر می‌کند. و بالعکس، گوش دادن به موسیقی هنگام فعالیت‌های بدنی موجب می شود که آهنگ فعالیت های انسان بیشتر شده و فعالیت مورد نظر را سریع‌تر انجام‌دهد.
و یا در مواقعی که برای انسان استرس زا است گوش دادن به موسیقی باعث کاهش 26 درصدی اظطراب انسان می‌شود.
\cite{lesiuk2005effect,siedliecki2006effect,roballey1985effect,anshel1978effect,pelletier2004effect}
\section{در‌صنایع ‌مختلف}\label{sec2}
موسیقی علاوه بر زندگی روزمره در صنایع مختلفی مانند فیلم سازی، تبلیغات ، درمان و دارویی نیز کاربرد دارد. در ادامه به کاربرد های آن در صنایع مختلف می پردازیم.
\begin{enumerate}
\item  {تبلیغات}
\item { فیلم سازی}
\item  {درمانی}
\subsection{تبلیغات}
\end{enumerate}
  آشنایی با موسیقی پس‌زمینه بیشتر بر نگرش‌ها نسبت به تبلیغات،  نگرش‌های برند و قصد خرید در شرایط مشارکت کم تأثیر می‌گذارد،  در حالی که تناسب محصول با موسیقی پس‌زمینه بیشتر در شرایط مشارکت بالا تأثیرگذار است.  این نشان می‌دهد که ویژگی‌های مؤثر موسیقی پس‌زمینه می‌تواند با توجه به مشارکت مصرف‌کننده متفاوت باشد. 
اگرچه موسیقی در تبلیغات می تواند تأثیر بسیار مثبتی بر مصرف کننده داشته باشد، اما
نگرش نسبت به محصول تجاری و نام تجاری نیز بسیار می‌تواند متقاوت باشد. 
مشکلات هنگام استفاده از موسیقی همیشه بخشی از مردم هستند که
از موسیقی ای که در یک آگهی بازرگانی پخش می شود، خوششان نمی آید، خواه به خاطر آن باشد که 
هنرمند،  ژانر یا سبک را دوست ندارند.  این را نمی توان کنترل کرد زیراکه این
یک عامل ذهنی بسته به شخصیت و ویژگی های فردی است. 
علاوه بر این، موسیقی همچنین می تواند به عنوان یک حواس پرتی و محدودیت برای مصرف کنندگان عمل کند
که سعی در پردازش اطلاعات ارائه شده در آگهی دارند. 
\cite{hee2014attributes,hoeberichts2012music}
\subsection{فیلم سازی}
در این بخش نتایج مطالعه افراد بر روی تاثیر یک موسیقی‌فیلم بر روی‌احساسات‌بیننده را مشاهده میکنیم.  چهار جفت عواطف قطبی تعریف شده در مدل پلاچیک به عنوان ویژگی های عاطفی اساسی مورد استفاده قرار گرفتند: شادی-غم،  انتظار- تعجب،  ترس-خشم و اعتماد-انزجار.  در مطالعه اولیه، هشت سکانس فیلم و هشت تم موسیقی به عنوان بهترین نمایندگان هر هشت احساس پلاچیک انتخاب شدند.  در آزمایش اصلی،  شرکت‌کنندگان کیفیت‌های احساسی ترکیب‌های فیلم-موسیقی را در هشت مقیاس هفت درجه‌ای قضاوت کردند.  نیمی از ترکیب ها همخوان بودند (مثلاً فیلم شاد - موسیقی شاد) و نیمی ناسازگار بودند (مثلاً فیلم شاد - موسیقی غمگین).  نتایج نشان داده است که اطلاعات بصری (فیلم) تأثیر بیشتری بر ارزیابی هیجان نسبت به اطلاعات شنیداری (موسیقی) دارد.  اثرات مدل سازی  پس‌زمینه موسیقی به کیفیت‌های احساسی بستگی دارد.  در برخی از ترکیب‌های نامتجانس (شادی-غم) تعدیل‌هایی در جهت‌های مورد انتظار به دست آمد (مثلاً موسیقی شاد غم یک فیلم غمگین را کاهش می‌دهد)، در برخی موارد (خشم-ترس) هیچ اثر مدل سازی به دست نیامد و در برخی موارد (اعتماد). -انزجار، انتظار-غافلگیری) اثرات مدل سازی  در جهت غیرمنتظره بود (مثلاً موسیقی قابل اعتماد ارزیابی انزجار از یک فیلم منزجر کننده را افزایش داد). این نتایج نشان می‌دهد که ارزیابی تأثیرات مشترک احساسات به رسانه (فیلم موسیقی را می‌پوشاند ) و کیفیت احساسی (سه نوع اثر مدل سازی ) بستگی دارد. 
دو آزمایش تأثیر موسیقی فیلم را بر اقناع روایت بررسی کردند.  در آزمایش اول،  شرکت‌کنندگان یک فیلم کوتاه را با موسیقی متن اصلی آن یا با موسیقی متن بی‌صدا تماشا کردند. در آزمایش دوم،  موسیقی متن موسیقی به فیلمی اضافه شد که در ابتدا بدون موسیقی تولید شده بود.  یافته‌ها حاکی از آن بود که شرکت‌کنندگان هنگام ارائه موسیقی متن فیلم، انتقال بیشتر به فیلم و توافق بیشتر با باورهای مرتبط با فیلم را گزارش کردند، اما تنها زمانی که موسیقی با لحن عاطفی فیلم مطابقت داشت. در حمایت از مدل حمل و نقل-تصویر
\LTRfootnote{
Green and Brock transportation theory
}
 تأثیر موسیقی فیلم بر باورها با انتقال به فیلم واسطه شد.
\cite{costabile2013effects,pavlovic2011effect}
\subsection{درمانی}
در این بخش نتایج مطالعه افراد بر روی تأثیر گوش دادن به موسیقی ترجیحی بر ادراک اضطراب و درد بیماران تحت همودیالیز را مشاهده میکنیم. در انجام‍این آزمایش‍از دو‌گروه‌از افراد‌استفاده‌شده‌است.  شصت نفر با تشخیص نارسایی کلیوی مرحله نهایی تحت درمان همودیالیز در این مطالعه شرکت کردند.  گوش دادن به موسیقی ترجیحی به عنوان مداخله استفاده شده است.  اضطراب و درد در پیش آزمون و پس آزمون اندازه گیری شده است. گروه کنترل در مرحله پس آزمون در اضطراب حالت به طور معنی داری نمره بالاتری نسبت به گروه آزمایش شده کسب کردند و شدت درد بالاتری را به طور معنی داری‌در مرحله پس آزمون تجربه کردند . 
این مطالعه نشان داد که گوش دادن به موسیقی در حین کشیدن غده بدخیم سوم نهفته فک پایین باعث سرکوب فعالیت اعصاب سمپاتیک 
\LTRfootnote{
Sympathetic nervous system
}

 در حین برش، برداشتن استخوان و جدا شدن تاج دندان می شود و اضطراب پس از درمان را تسکین می دهد.  مطالعات آتی بر روی مکانیسم‌های درگیر و روش‌های پیشگیری از شروع حوادث سیستمیک تمرکز خواهند کرد.
نمره پیش آزمون نشان داد که اکثر بیماران دارای سطح متوسط ​​(5/77 درصد) و 15 درصد دارای سطح اضطراب حالت شدید بودند. اندازه گیری مکرر نشان داد که میانگین نمره اضطراب حالت در هر یک از سه اندازه گیری پس آزمون در گروه آزمایش به طور قابل توجهی کمتر از گروه کنترل بود.  میانگین تفاوت بین گروه آزمایش و کنترل مقدار‌معنا داری است.
\cite{li2012effects,yamashita2019effects,pothoulaki2008investigation}
\section{خلاصه}
با توجه به مطالب بالا ونتایج به دست آمده مشاهده میکنیم که موسیقی بخش عظیمی از زندگی مارا تشکیل می دهد.  و در اکثرفعالیت های ما تاثیر بسزایی دارد ، و گوش دادن به آن باعث بهبود فعالیت های روزانه می شود. 
حال اگر بتوان این موسیقی را به نوعی مختص هر فعالیت با استفاده از هوش مصنوعی رایانه ای گسترش داد می توان تاثیر مثبت آن را بیشتر کرد. در نتیجه انسان ها هنگامی گوش دادن به موسیقی می توانند بازدهی بیشتر، درد کمتر ، آرامش بیشتر و در کل حس ها را بهتر دریافت کرده و روز مثبتی داشته باشند. 
تحلیل هر موسیقی متناسب با مکان و زمان آن یکی از چالش هایی است که با توجه به بخش 
\ref{sec2}.
می توان به آن پی برد. در بخش بعدی به مطالعه علم هوش مصنوعی رایانه ای می پردازیم تا بتوانیم این چالش را با استفاده از رایانه حل کنیم.
