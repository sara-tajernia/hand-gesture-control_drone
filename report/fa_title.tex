%% -!TEX root = AUTthesis.tex
% در این فایل، عنوان پایان‌نامه، مشخصات خود، متن تقدیمی‌، ستایش، سپاس‌گزاری و چکیده پایان‌نامه را به فارسی، وارد کنید.
% توجه داشته باشید که جدول حاوی مشخصات پروژه/پایان‌نامه/رساله و همچنین، مشخصات داخل آن، به طور خودکار، درج می‌شود.
%%%%%%%%%%%%%%%%%%%%%%%%%%%%%%%%%%%%
% دانشکده، آموزشکده و یا پژوهشکده  خود را وارد کنید
\faculty{دانشکده مهندسی کامپیوتر}
% گرایش و گروه آموزشی خود را وارد کنید
\department{}
% عنوان پایان‌نامه را وارد کنید
\fatitle{ارزیابی ترکیب شبکه‌های عصبی گرافی و مدل پوینت‌نت برای پردازش داده‌‌های گرافی و ابرنقاط}
% نام استاد(ان) راهنما را وارد کنید
\firstsupervisor{دکتر مهدی جوانمردی }
\secondsupervisor{}
% نام استاد(دان) مشاور را وارد کنید. چنانچه استاد مشاور ندارید، دستور پایین را غیرفعال کنید.
%\firstadvisor{نام کامل استاد مشاور}
%\secondadvisor{استاد مشاور دوم}
% نام نویسنده را وارد کنید
\name{محمد}
% نام خانوادگی نویسنده را وارد کنید
\surname{چوپان}
%%%%%%%%%%%%%%%%%%%%%%%%%%%%%%%%%%
\thesisdate{اردیبهشت ۱۴۰3}

% چکیده پایان‌نامه را وارد کنید
\fa-abstract{این پایان‌نامه به اکتشاف یادگیری عمیق و داده‌های ابرنقاط و گرافی می‌پردازد و نحوه عملکرد معماری‌های متفاوت در دسته‌بندی و پردازش این دو داده را ارزیابی می‌کند.
ابتدا ما به بررسی خودرو‌های خودران و اهمیت آن‌ها در زندگی روزمره می‌پردازیم.
سپس معرفی داده‌های ابرنقاط 
\LTRfootnote{
Point Cloud Data
}
و گرافی و کاربرد آن‌ها در زمینه خودرو‌های خودران را بررسی می‌کنیم و مجموع داده‌های استفاده شده در این پروژه را معرفی می‌کنیم.
پس از آن بررسی معماری های تجمیع گراف خود‌توجه
\LTRfootnote{
Self Attention Graph Pooling
}
و پوینت نت 
\LTRfootnote{
Point NeT
}
می‌پردازیم.
در قدم بعدی معرفی معماری‌های حاصل از ترکیب این دو و نتایج حاصل از دو معماری به صورت مجزا و در نهایت نتایج به دست آمده از ترکیب این دو معماری را بررسی می‌کنیم.
در قدم اخر نیز تاثیر هر یک از نقاط داده‌های خود را به صورت جدا بررسی کرده و بهبود‌های داده شده در عملکرد معماری معرفی شده را با استفاده از نتایج شهودی بررسی خواهیم کرد.
}


% کلمات کلیدی پایان‌نامه را وارد کنید
\keywords{یادگیری عمیق، ابرنقاط، معماری تجمیع خودتوجه، پوینت نت، خودرو‌های خودران}



\AUTtitle{a}
%%%%%%%%%%%%%%%%%%%%%%%%%%%%%%%%%%
\vspace*{7cm}
\thispagestyle{empty}
\begin{center}
\includegraphics[height=5cm,width=12cm]{besm}
\end{center}