\chapter{مقدمه}
% \afterpage{\newpage}
موسیقی در زندگی انسان نقش حیاتی را دارا است. علاوه بر زندگی روزمزه انسان در صنعت هایی مانند فیلم سازی، تبلیغات، درمان و ... کاربر های متفاوتی دارد. 
حال ساخت یک موسیقی دل نشین و کاربردی چالشی است که امروزه اکثر هنرمندان در این زمینه با آن درگیرند. 
در این پژوهش نگاهی می‌اندازیم بر بررسی و ساخت موسیقی با استفاده از هوش مصنوعی، بررسی چالش ها و روش های متفاوت انجام این کار و در انتها  بر تاثیر موسیقی‌های ساخته شده بر روی صنایع متفاوت نگاهی می‌اندازیم. 
در فصل ابتدایی به بررسی موسیقی و تاثیر آن بر روی زندگی روزمره می‌پردازیم. سپس در فصل دوم به بررسی هوش مصنوعی و الگوریتم ها متفاوت آن می‌پردازیم. در فصل سوم نیز نگاهی میکنیم به نحوه تبدیل موسیقی به زبان رایانه و برعکس آن،سپس نحوه ساخت موسیقی با استفاده از هوش مصنوعی را بررسی میکنیم. در فصل چهارم نگاهی می‌اندازیم بر موسیقی‌های ساخته شده با استفاده از هوش مصنوعی ، در آخر نیز نتیجه از مقایسه موسیقی‌های ساخته شده با این روش و موسیقی‌های نواخته شده و چشم اندازی از آینده این موسیقی‎ها می‎آوریم. 