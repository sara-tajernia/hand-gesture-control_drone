\chapter{ نتيجه‌گيري و پیشنهادات}



% \afterpage{\newpage}
%%%%%%%%%%%%%%%%%%%%%%%%%%%%%%%%%%%%%%%%%%%
\section{نتیجه گیری}
 در این مطالعه، ما سعی کردیم با توسعه یک رویکرد، رفتار تولید موسیقی را بررسی کنیم. 
متشکل از شش مرحله، یعنی تبدیل فرمت به فرمت آهنگ، رمزگذاری، تقویت،
پنجره سازی، یادگیری با استفاده از شبکه عصبی و تولید موسیقی. آزمایش های مختلفی با
پارامترهای مختلف برای بررسی بهبود یادگیری نتایج ایجاد شده توسط شبکه ما بود
از منظر موسیقی خیلی چشمگیر نیست، اما بد هم نیست. مدل ما یک درک اساسی از
ریتم و هارمونی قطعات تولید شده ساختار بزرگی ندارند. آغاز، پایان و موتیف
تکرار برخی از این موارد می‌توانست توسط شبکه‌ای شبیه به مدل والدر مورد توجه قرار گیرد.
برخی از مشکلاتی که حل نشده باقی می‌مانند این است که مدل در موارد خاص چند نت خوب را خروجی می‌دهد. 
سپس تمام صفرها همچنین گاهی اوقات ملودی های کوتاهی تولید می کند که بی نهایت تکرار می شوند. این دو مشکل نیاز دارند
به طور عمیق مورد بررسی قرار گیرد و ممکن است یک فرآیند پست برای جلوگیری از وقوع چنین مواردی به ندرت اجرا شود
موارد اتفاق افتاده مشکل سوم این بود که تابع از دست دادن بیت 89 
  \verb;a;
  بودن را در نظر نمی گیرد
ارزش شناور اگر دو مشکل اول را حذف کنیم زیرا همیشه رخ نمی دهند، عملکرد مدل
از نظر موسیقی قابل قبول است و با صدای کمتر هارمونی خوبی می دهد. برای مشکل سوم، ساختار
 باید با اتصال مستقیم ورودی به خروجی به بیت 89 اهمیت بیشتری بدهد.
می تواند تأثیر واضحی بر شبکه ‌عصبی و در تابع ضرر داشته باشد.
کار احتمالی آینده برای کار ما می تواند استفاده از توابع از دست دادن سفارشی باشد که پراکندگی را بهتر مدل می کند. 
ماهیت بردارها همچنین می‌توانیم از یک بیت 90 اضافی برای نمایش موقعیت درون قطعه موسیقی استفاده کنیم.
که می تواند به معرفی ساختار بزرگتر در قطعات تولید شده کمک کند. یکی دیگر از بهبودهای احتمالی
استفاده از مدل هایی غیر از این مانند شبکه های متخاصم مولد. ما همچنین می توانیم
با انواع دیگر مجموعه داده های موسیقی با ویژگی های متفاوت با مجموعه ای که ما استفاده کردیم آزمایش کنیم. بالاخره مشکل
می تواند به فایل های چند ابزاری و تولید چند ابزاری گسترش یابد. امیدوارم این ها امکان پذیر باشد
پیشرفت‌ها می‌تواند الهام‌بخش آزمایش‌های بیشتر توسط دیگران باشد.
هر چند که شنیدن موسیقی زنده لذت بسیار زیادی را دارد و تا آینده‌ای دور نمی‌توان جای آن را با موسیقی تولید شده  با رایانه گرفت. اما شاید در آینده‌ای دور این روش جایگزین انسان ها شود این روش بسیار دلنشین تر برای انسان ‌های نسل جدید باشد. 
\section{پیشنهادات}
با استفاده از موسیقی‌های تولید شده توسط هوش مصنوعی می‌توان موسیقی را در صنایع مختلف گسترش داد. برای مثال اگر رفتار مشتریان  در یک فروشگاه را تحلیل کرد و با استفاده از آن ها موسیقی های جدید تولید کرد که به سمت هدف خاصی در خرید مشتریان را هدایت کرد. در موردی دیگر در فیلم ها می‌توان با توجه به اینکه احساسات بیننده ها در صحنه‌ای از فیلم باید چه شکلی باشد موسیقی مناسب تری را تولید کرد حتی با استفاده از پالس‌های عصبی افراد می‌توان موسیقی مناسب برای هر فرد را تولید کرده و برای بیماری‌های خاصی از آن استفاده کرد. 