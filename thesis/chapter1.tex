\chapter{کار‌های مشابه}
\section{مقدمه}
در این فصل هدف ما بررسی پروژه های مشابه است تا بتوان از آنها در روند پروژه کمک گرفت. همچنین در این راه می‌توان با توجه به نتایج و ارزیابی پروژه‌های دیگر بستری را فراهم کرد تا نتیجه پروژه را با دیگر کارهای مشابه مقایسه کرد.
\\
به صورت کلی پروژه‌هایی با هدف کنترل پهپاد با ژست دست در 2 دسته قرار می‌گیرند.
\begin{itemize}
    \item کنترل پهپاد با کمک بینایی ماشین که شامل شبکه‌هایی برای پردازش تصویر است. 
    \item کنترل پهپاد با دستکش‌های سنسور دار از جمله سنسور \lr{IMU} که نیازمند سخت‌افزار خاص برای پیدا کردن موقعیت نقاط دست است. مانند پروژه‌های \lr{Motion Estimation and Hand Gesture Recognition-Based Human–UAV Interaction Approach in Real Time} \cite{yoo2022motion} و \lr{Hand gesture recognition with convolutional neural networks for the multimodal UAV control} \cite{ma2017hand}.
    \item وجود دستگاه کنترل کننده حرکت جهشی\lr{Leap Motion Controller} که با توجه آن ویژگی‌های دست با دقت بالا اندازه گیری شده و با کمک شبکه‌های عصبی ژست دست تشخیص داده میشود. پروژه‌ی
    \lr{Deep Learning Based Hand Gesture Recognition and UAV Flight Controls} \cite{hu2020deep} و \lr{Gesture control of drone using a motion controller} \cite{sarkar2016gesture} نمونه‌ای از این جمله پروژه‌ها هستند. 
\end{itemize}

از بین این موارد پروژه ما مربوط به اولین گزینه است که تنها سخت‌افزار مورد نیاز به جز پهپاد دوربین نصب شده روی پهپاد است. که به بررسی نمونه‌ی این پروژه‌ها می‌پردازیم.

% \section{کارهای مشابه}

% \subsection{مقاله \lr{Deep Learning Based Hand Gesture Recognition and UAV Flight Controls}}
% \section{مقاله \lr{Hand Gesture Controlled Drones: An Open Source Library}}
% این پروژه بر پیاده سازی یک سیستم کنترل برای هواپیماهای بدون سرنشین با استفاده از حرکات دست، مشابه رویکرد مورد بحث در مقاله تمرکز دارد. هدف آن استفاده از شبکه‌های عصبی یادگیری عمیق برای تشخیص لحظه‌ای حرکات دست پویا برای کنترل پرواز پهپاد است.
% \subsection{روش‌شناسی}
% پروژه پیاده‌سازی شده حاوی طراحی و آموزش شبکه‌های عصبی در راستای حرکت پهپاد می‌باشد. این سیستم شامل پیش پردازش داده‌ها، انتخاب ویژگی، ماژول شبکه عصبی یادگیری عمیق برای تشخیص ژست و ماژول کنترل پهپاد برای ترجمه ژست‌های شناسایی شده به دستورات حرکت پهپاد است.
% \\
% برای شناسایی و تشخیص ژست‌های دست، ابتدا از یک شبکه عصبی برای تشخیص موقعیت دست استفاده می‌شود. این شبکه با استفاده از داده‌های ورودی به‌صورت تصویری، موقعیت دقیق دست را در تصویر تعیین می‌کند. پس از تشخیص موقعیت دست، ویژگی‌های  برای استخراج ویژگی‌های مهم از تصویر استفاده می‌شود.
% \\
% ویژگی‌های \lr{Haar} مجموعه‌ای از الگوریتم‌های تشخیص ویژگی است که به‌صورت یکپارچه و متناوب از تصاویر استفاده می‌کنند تا ویژگی‌های خاصی از تصویر را شناسایی کنند. این ویژگی‌ها شامل اندازه‌ها و الگوهای مختلفی از رنگ و شدت نور می‌شوند. به‌عنوان مثال، ویژگی‌های \lr{Haar} می‌توانند مرزهای قابل توجه و سایه‌ها را شناسایی کنند که برای تشخیص اشیاء مهم است.
% \\
% در نهایت، ویژگی‌های مختلفی که از تصویر استخراج شده به‌عنوان ورودی به ماژول شبکه عصبی یادگیری عمیق برای تشخیص ژست دست مورد استفاده قرار می‌گیرد. و مدل مورد نظر ساخته می‌شود.

% \subsection{نتیجه بدست آمده}
% این پروژه دقت بالایی در تشخیص ژست دست و کنترل پرواز پهپاد دست می یابد. برای خروجی این پروژه ۵ حالت دست مدنظر قرار گرفته‌اند. 
% دقت متوسط تقریباً 471.97 درصد است که عملکرد عالی را نشان می‌دهد. البته قابل ذکر است که این دقت در پس‌زمینه‌های بهم ریخته و همچنین در شرایط نوری مختلف بسیار متغیر است چرا که ویژگی \lr{Haar} به سایه و رنگ‌های درون تصویر بسیار حساس است \cite{natarajan2018hand}.



\section{مقاله \lr{Hand Gesture Controlled Drones: An Open Source Library} و \lr{Hand Gestures For Drone Control Using Deep Learning}}
این دو پروژه هر دو یک هدف دارند و با یک ساختار و معماری به آن رسیده‌اند. آنها بر روی پیاده سازی یک سیستم کنترل برای هواپیماهای بدون سرنشین با استفاده از حرکات دست، مشابه رویکرد مورد بحث در مقاله تمرکز دارند. هدف آنها استفاده از شبکه‌های عصبی یادگیری عمیق برای تشخیص لحظه‌ای حرکات دست پویا برای کنترل پرواز پهپاد است.
\subsection{روش‌شناسی}
پروژه‌های پیاده‌سازی شده حاوی طراحی و آموزش شبکه‌های عصبی در راستای حرکت پهپاد می‌باشند. این سیستم شامل پیش پردازش داده‌ها، انتخاب ویژگی، ماژول شبکه عصبی یادگیری عمیق برای تشخیص ژست و ماژول کنترل پهپاد برای ترجمه ژست‌های شناسایی شده به دستورات حرکت پهپاد است.
\\
برای شناسایی و تشخیص ژست‌های دست، ابتدا از یک شبکه عصبی برای تشخیص موقعیت دست استفاده می‌شود. این شبکه با استفاده از داده‌های ورودی به‌صورت تصویری، موقعیت دقیق دست را در تصویر تعیین می‌کند. پس از تشخیص موقعیت دست، ویژگی‌های  برای استخراج ویژگی‌های مهم از تصویر استفاده می‌شود.
\\
ویژگی‌های \lr{Haar} مجموعه‌ای از الگوریتم‌های تشخیص ویژگی است که به‌صورت یکپارچه و متناوب از تصاویر استفاده می‌کنند تا ویژگی‌های خاصی از تصویر را شناسایی کنند. این ویژگی‌ها شامل اندازه‌ها و الگوهای مختلفی از رنگ و شدت نور می‌شوند. به‌عنوان مثال، ویژگی‌های \lr{Haar} می‌توانند مرزهای قابل توجه و سایه‌ها را شناسایی کنند که برای تشخیص اشیاء مهم است.
\\
در نهایت، ویژگی‌های مختلفی که از تصویر استخراج شده به‌عنوان ورودی به ماژول شبکه عصبی یادگیری عمیق برای تشخیص ژست دست مورد استفاده قرار می‌گیرد. و مدل مورد نظر ساخته می‌شود.

\subsection{نتیجه}
این پروژه‌ها دقت بالایی را در تشخیص ژست دست و کنترل پرواز پهپاد به‌دست آورده‌اند. به عنوان خروجی پروژه \lr{Hand Gesture Controlled Drones: An Open Source Library} ۵ حالت دست را مدنظر قرار گرفته‌. 
 دقت متوسط این مدل برابر 471.97 درصد است که عملکرد عالی را نشان می‌دهد. همچنین در پروژه \lr{Hand Gestures For Drone Control Using Deep Learning} ۹ حالت دست مدنظر قرار گرفته‌شده و دقت آن برابر ۳.۸۳ درصد است.
 قابل ذکر است که این دقت در پس‌زمینه‌های بهم ریخته و همچنین در شرایط نوری مختلف بسیار متغیر است چرا که ویژگی \lr{Haar} به سایه و رنگ‌های درون تصویر بسیار حساس اند \cite{natarajan2018hand} \cite{hadri2018hand}.



\section{مقاله \lr{MediaPipe Hands: On-device Real-time Hand Tracking } و \lr{Applying Hand Gesture Recognition for User Guide Application Using MediaPipe}}
در این مقاله‌ها از کتابخانه \lr{MediaPipe} استفاده‌ شده‌است تا بتوان ۲۱ نقطه عطف دست را پیدا کرده و در پروژه‌های گوناگون از جمله پیدا کردن ژست دست و افکت‌های \lr{AR} استفاده کند. ما در پروژه خود از این کتابخانه استفاده می‌کنیم تا بتوانیم مدلی سبک و ساده پیاده کنیم.

\subsection{روش‌شناسی}
 در این مقاله به یررسی این کتابخانه پرداخته‌شده و ما از آن استفاده می‌کنیم تا بتوانیم مدلی برای پیش‌بینی ژست دست استفاده کنیم.
برای پیاده‌سازی این پروژه از دو شبکه کانولوشن استفاده شده است. شبکه اول که برای پیدا کردن کف دست در تصویر استفاده می‌شود و شبکه دوم که به عنوان ورودی موقعیت عکس دست پیدا شده را دریافت و مختصات ۲۱ تقطه عطف را موقعیت‌یابی می‌کند.


% \section{مقاله \lr{MediaPipe Hands: On-device Real-time Hand Tracking } و \lr{Applying Hand Gesture Recognition for User Guide Application Using MediaPipe}}
% در این مقاله از کتابخانه \lr{MediaPipe} استفاده‌ شده‌است. بدین صورت که ابتدا موقعیت دست را پیدا کرده، سپس مختصات ۲۱ تقطه عطف را موقعیت‌یابی می‌کند. در این مقاله به یررسی این کتابخانه پرداخته‌شده و ما از آن استفاده می‌کنیم تا بتوانیم مدلی برای پیش‌بینی ژست دست استفاده کنیم که ساده و سبک‌ باشد.

% \subsection{روش‌شناسی}
% برای پیاده‌سازی این پروژه از دو شبکه کانولوشن استفاده شده است. شبکه اول که برای پیدا کردن کف دست در تصویر استفاده می‌شود

% \subsection{روش‌شناسی}
% برای پیاده‌سازی این پروژه از دو شبکه کانولوشن استفاده شده است. شبکه اول که برای پیدا کردن کف دست در تصویر استفاده می‌شود


\subsection{نتیجه}
مدل‌های طراحی شده در این مقالات برای تشخیص نقاط عطف دست از دقت ۷.۹۵ درصد برخوردار هستند که دقت بسیار بالایی محاسبه می‌شود. همچنین این مدل به نور و تصویر پس‌زمینه وابسته نیست و دقت متوسط آن در زمینه‌های مختلف اندازه‌گیری شده لذا مدل را کاربردی‌ و مورد پسندتر می‌کند\cite{zhang2020mediapipe} .
\section{جمع‌بندی}
پروژه‌های مشابه با کار ما که با کمک پینایی ماشین پهپاد را کنترل می‌‌کنند به 4 دسته کلی تفکیک می‌شوند.
\begin{enumerate}
    \item  پیاده‌سازی با کمک کلاس \lr{MediaPipe} برای تشخیص نقاط عطف دست و شبکه‌ای برای تشخیص ژست دست با کمک نقاط عطف دست.
    \item استفاده از ویژگی‌های \lr{Haar} و پیدا کردن ژست دست توسط آنها.
    \item استخراج ویژگی‌های تصویر از جمله پارامترهایی مانند زاویه انحراف، مختصات، قدرت گرفتن دست و استفاده آنها در شبکه برای رسیدن به کلاس ژست دست.
    \item تشخیص دست\LTRfootnote{Hand detection} برای پیدا کردن موقعیت دست در هر فریم تصویر و استفاده از آن به هر پیکسل \lr{RGB} و کلاس‌بندی ژست دست با توجه به تصویر.
\end{enumerate}