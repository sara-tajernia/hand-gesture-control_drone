%% -!TEX root = AUTthesis.tex
% در این فایل، عنوان پایان‌نامه، مشخصات خود، متن تقدیمی‌، ستایش، سپاس‌گزاری و چکیده پایان‌نامه را به فارسی، وارد کنید.
% توجه داشته باشید که جدول حاوی مشخصات پروژه/پایان‌نامه/رساله و همچنین، مشخصات داخل آن، به طور خودکار، درج می‌شود.
%%%%%%%%%%%%%%%%%%%%%%%%%%%%%%%%%%%%
% دانشکده، آموزشکده و یا پژوهشکده  خود را وارد کنید
\faculty{دانشکده مهندسی کامپیوتر}
% گرایش و گروه آموزشی خود را وارد کنید
\department{}
% عنوان پایان‌نامه را وارد کنید
\fatitle{هدایت پهپاد با علائم دست مبتنی بر بینایی ماشین}
% نام استاد(ان) راهنما را وارد کنید
\firstsupervisor{دکتر مهدی جوانمردی }
\secondsupervisor{}
% نام استاد(دان) مشاور را وارد کنید. چنانچه استاد مشاور ندارید، دستور پایین را غیرفعال کنید.
%\firstadvisor{نام کامل استاد مشاور}
%\secondadvisor{استاد مشاور دوم}
% نام نویسنده را وارد کنید
\name{سارا }
% نام خانوادگی نویسنده را وارد کنید
\surname{تاجرنیا}
%%%%%%%%%%%%%%%%%%%%%%%%%%%%%%%%%%
\thesisdate{خرداد ۱۴۰۳}


% چکیده پایان‌نامه را وارد کنید
\fa-abstract{پهپادهای تجاری که به عنوان هواپیما‌های بدون سرنشین\LTRfootnote{Unmanned Aerial Vehicles} 
نیز شناخته می‌شوند، به سرعت در حال رایج شدن هستند. این پهپاد‌ها در زمینه‌های مختلف مانند نظارت بر رویدادهای ورزشی، حمل و نقل تجهیزات
 و کالاهای اضطراری، فیلمبرداری، عکس‌برداری هوایی و ... مورد استفاده قرار می‌گیرند.
  \\
هدف این پروژه توسعه سیستمی است که بتوان با استفاده از آن, از حرکات دست به عنوان روشی برای کنترل پرواز پهپاد‌ها استفاده کرد.
بدین صورت که با استفاده از روش‌های مبتنی بر بینایی ماشین\LTRfootnote{Computer Vision}، روشی بصری برای ارتباط بدون واسته کنترلر, بین پهپاد و اپراتور آن ایجاد کرد.
 روش‌های مبتنی بر بینایی ماشین با استفاده از دوربین هواپیما‌های بدون سرنشین که تصاویر اطراف را گرفته و پس
 از تحلیل تصاویر و تشخیص الگوی دست، اطلاعات معناداری از آن را استخراج می‌کنند. ساختار این پروژه از دو ماژول
 اصلی تشخیص حرکت دست\LTRfootnote{Hand Detection}
 و دستور به هواپیمای بدون سرنشین تشکیل شده است. برای ماژول اول از روش مبتنی بر
 یادگیری عمیق\LTRfootnote{Deep Learning}
 استفاده شده است تا بتوان با دقت بالا علامت دست را پیش‌بینی کرد. ماژول دوم نیز وظیفه ارتباط با پهپاد را بر عهده دارد که نتیجه تشخیص را برای پهپاد ارسال می‌کند تا از آن پیروی کند.
 در پروژه پیاده‌سازی شده نتایج عددی به دست آمده دقت بالای ۹۸ درصد را در تشخیص درست علائم دست نشان می‌دهند، همچنین پیاده‌سازی میدانی این پروژه بر روی پهپاد دی‌جی‌آی تلو گواه بر عملکرد موفقیت‌آمیز این سیستم را دارد.
 }
% کلمات کلیدی پایان‌نامه را وارد کنید
\keywords{
    پهپاد، علامت دست، شبکه‌‌‌ عصبی پیچشی\LTRfootnote{Convolutional Neural Network(CNN)}
    ، شبکه‌‌‌ حافظه طولانی کوتاه مدت\LTRfootnote{Long Short-Term Memory(LSTM)}، شبکه عصبی بازگشتی\LTRfootnote{Recurrent Neural Network(RNN)}، رابط انسان و پهپاد\LTRfootnote{Human–Drone Interface}
}



\AUTtitle{a}
%%%%%%%%%%%%%%%%%%%%%%%%%%%%%%%%%%
\vspace*{7cm}
\thispagestyle{empty}
\begin{center}
\includegraphics[height=5cm,width=12cm]{besm}
\end{center}