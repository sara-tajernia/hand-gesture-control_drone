%% -!TEX root = AUTthesis.tex
% در این فایل، عنوان پایان‌نامه، مشخصات خود، متن تقدیمی‌، ستایش، سپاس‌گزاری و چکیده پایان‌نامه را به فارسی، وارد کنید.
% توجه داشته باشید که جدول حاوی مشخصات پروژه/پایان‌نامه/رساله و همچنین، مشخصات داخل آن، به طور خودکار، درج می‌شود.
%%%%%%%%%%%%%%%%%%%%%%%%%%%%%%%%%%%%
% دانشکده، آموزشکده و یا پژوهشکده  خود را وارد کنید
\faculty{دانشکده مهندسی کامپیوتر}
% گرایش و گروه آموزشی خود را وارد کنید
\department{}
% عنوان پایان‌نامه را وارد کنید
\fatitle{هدایت پهپاد با علائم دست مبتنی بر بینایی ماشین}
% نام استاد(ان) راهنما را وارد کنید
\firstsupervisor{دکتر مهدی جوانمردی }
\secondsupervisor{}
% نام استاد(دان) مشاور را وارد کنید. چنانچه استاد مشاور ندارید، دستور پایین را غیرفعال کنید.
%\firstadvisor{نام کامل استاد مشاور}
%\secondadvisor{استاد مشاور دوم}
% نام نویسنده را وارد کنید
\name{سارا }
% نام خانوادگی نویسنده را وارد کنید
\surname{تاجرنیا}
%%%%%%%%%%%%%%%%%%%%%%%%%%%%%%%%%%
\thesisdate{اردیبهشت ۱۴۰3}


% چکیده پایان‌نامه را وارد کنید
\fa-abstract{پهپادهای تجاری که به عنوان هواپیما‌های بدون سرنشین\LTRfootnote{Unmanned aerial vehicles} 
نیز شناخته می‌شوند، به سرعت در حال رایج شدن هستند و در بسیاری از کاربردهای مختلف مانند نظارت برای رویدادهای ورزشی، حمل و نقل تجهیزات
 و کالاهای اضطراری، فیلمبرداری، عکاسی هوایی و بسیاری از فعالیت‌های دیگر مورد استفاده قرار می‌گیرند.
  \\
هدف این پروژه توسعه سیستمی است که از حرکات دست به عنوان روشی برای کنترل پرواز پهپاد استفاده شود.
بدین صورت که با استفاده از روش‌های بینایی ماشین\LTRfootnote{Computer vision}، روشی بصری برای ارتباط بدون عامل بین پهپاد و اپراتور آن ایجاد می‌کنیم.
 روش‌های مبتنی بر بینایی ماشین بر توانایی دوربین هواپیماهای بدون سرنشین متکی هستند. بدین صورت که تصاویر اطراف را گرفته و با استفاده
 از  ترجمه تصاویر و تشخیص الگوی دست، اطلاعات معناداری را استخراج می‌کنند. ساختار این پروژه از دو ماژول
 اصلی تشکیل شده است: تشخیص حرکت دست\LTRfootnote{Hand detection}
 و دستور به هواپیمای بدون سرنشین. برای ماژول اول از یک روش
 یادگیری عمیق\LTRfootnote{Deep learning} استفاده شده است. الگوریتم‌ها و تکنیک‌های پردازش تصویر به‌ عنوان روشی پویا برای شناسایی ژست‌ها و
 حرکات دست معرفی شده‌اند. ماژول دوم وظیفه ارتباط با پهپاد را بر عهده دارد. بدین صورت که پیام‌های بین سیستم
 پیشنهادی و پهپاد متصل به سیستم را ارسال و دریافت می‌کند و طبق آن پیام‌ها عملیات مورد نظر را اجرا می‌کند.
  }


% کلمات کلیدی پایان‌نامه را وارد کنید
\keywords{
    پهپاد، هواپیمای بدون سرنشین، ژست دست، بینایی ماشین، شبکه‌های عصبی پیچشی\LTRfootnote{Convolutional neural network}
    ، حافظه طولانی کوتاه مدت\LTRfootnote{Long short-term memory}، یادگیری ماشین\LTRfootnote{Machine learning}
    ، رابط انسان و پهپاد\LTRfootnote{Human–drone interface}
}



\AUTtitle{a}
%%%%%%%%%%%%%%%%%%%%%%%%%%%%%%%%%%
\vspace*{7cm}
\thispagestyle{empty}
\begin{center}
\includegraphics[height=5cm,width=12cm]{besm}
\end{center}