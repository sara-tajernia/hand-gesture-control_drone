\chapter{هوش‌مصنوعی}
% \afterpage{\newpage}
\section{نگاهی‌برعلم‌هوش‌مصنوعی}
هوش مصنوعی‌ هوشی است که توسط
ماشین آلات در علوم کامپیوتر، زمینه تحقیقات هوش مصنوعی تعریف می شود. 
 عنوان مطالعه «عوامل هوشمند»: هر وسیله ای که
محیط خود را درک می کند و اقداماتی را انجام می دهد که  شانس موفقیت آن‌را در یک هدف به حداکثر می رساند. 
 اصطلاح عامیانه
"هوش مصنوعی" زمانی اعمال می شود که یک ماشین  کارکردهای "شناختی" که انسان با انسان دیگر مرتبط می کند تقلید کند. 
 توانایی ها
در حال حاضر به عنوان هوش‌مصنوعی طبقه بندی می شود که شامل درک موفقیت آمیزگفتار انسان است. ،
 رقابت در سطح بالایی در بازی استراتژیک، 
سیستم های (مانند شطرنج و گو)، ماشین های خودران و‌یا‌هوشمند
مسیریابی در شبکه های تحویل محتوا، شبیه سازی های نظامی و
تفسیر داده های پیچیده
مشکلات(یا اهداف)  باشند. اهداف بلند مدت تحقیقات هوش مصنوعی عبارتند از
استدلال، دانش، برنامه ریزی، یادگیری، زبان طبیعی
پردازش (ارتباط)، ادراک و توانایی
حرکت و دستکاری اشیا وهوش عمومی.  
هوش‌مصنوعی دارای دو روش برای تحلیل داده ها و یادگیری است که عبارتند از :
\begin{enumerate}
    \item{یادگیر‌ماشین }
    \item{یادگیری‌عمیق }
\end{enumerate}
\subsection{یادگیری‌ ماشین}
یادگیری ماشین زیرشاخه علوم کامپیوتر است که
به گفته آرتور ساموئل در سال 1959، "کامپیوترها
توانایی یادگیری بدون برنامه‌ریزی صریح را دارند."
از مطالعه تشخیص الگو و
نظریه یادگیری محاسباتی درهوش مصنوعی
، یادگیری ماشین مطالعه‌ و
ساخت الگوریتم هایی که می توانند از آنها یاد بگیرند و بسازند را ای را بررسی می کند. 
پیش‌بینی‌ها بر روی داده‌ها  - چنین الگوریتم‌هایی بر موارد زیر غلبه می‌کنند
دستورالعمل های برنامه کاملاً ثابت با ایجاد داده محور
پیش بینی ها یا تصمیم گیری ها، از طریق ساختن یک مدل از
ورودی های نمونه یادگیری ماشینی در طیف وسیعی از موارد استفاده می شود. 
وظایف محاسباتی که در آن طراحی و برنامه نویسی صریح است
الگوریتم هایی با عملکرد خوب دشوار یا غیرقابل اجرا هستند.
به عنوان مثال برنامه های کاربردی شامل فیلتر کردن ایمیل، تشخیص
متجاوزان شبکه یا خودی های مخرب که روی یک داده کار می کنند. 
نقض،  تشخیص کاراکتر نوری ،  یادگیری به
رتبه و بینایی کامپیوتری
یادگیری ماشینی ارتباط نزدیکی ویا (اغلب همپوشانی دارد)  با آمار محاسباتی، که همچنین بر روی
پیش بینی با استفاده از رایانه های قوی دارد
 ارتباط با بهینه سازی ریاضی، که روش ها را ارائه می دهد،
حوزه های تئوری و کاربرد در این زمینه. فراگیری ماشین
گاهی اوقات با داده کاوی ترکیب می شود،  که بیشتر دومی است
که در زیر زمینه بیشتر بر تجزیه و تحلیل داده های اکتشافی متمرکز است و می باشد
به عنوان یادگیری بدون نظارت شناخته می شود. یادگیری ماشین نیز می تواند
بدون نظارت  باشد و برای یادگیری و ایجاد خط پایه استفاده شود
نمایه های رفتاری برای موجودیت های مختلف و سپس برای یافتن آن ها  استفاده می شود. 
ناهنجاری های معنی دار
در حوزه تجزیه و تحلیل داده ها، یادگیری ماشینی یک
روش مورد استفاده برای ابداع مدل ها و الگوریتم های پیچیده که
خود را به پیش بینی وا‌میدارد.  در استفاده تجاری،
به عنوان تجزیه و تحلیل پیش بینی شناخته می شود. این مدل های تحلیلی اجازه می دهد
محققان، دانشمندان داده، مهندسان و تحلیلگران برای «تولید
تصمیمات و نتایج قابل اعتماد و قابل تکرار» و کشف « بینش پنهان».
 از طریق یادگیری از روابط تاریخی و
روند در داده ها به آن ها دست یابند.


\subsection{یادگیری‌عمیق
}

یادگیری عمیق (همچنین به عنوان یادگیری ساختار یافته عمیق ،
یادگیری سلسله مراتبی یا یادگیری ماشین عمیق  شناخته می شود). مطالعه‌ی
شبکه های عصبی مصنوعی و یادگیری ماشینی مرتبط
الگوریتم هایی که شامل بیش از یک لایه پنهان هستند است. نمونه هایی از شبکه های عمیق:
\begin{itemize}
\item از یک آبشار چند لایه غیر‌برای برای واحد های پردازشی استفاده میکند. 
تا بتواند خروجی لایه های قابل را به بعدی برساند. 
الگوریتم ما ممکن است نظارت شده، یا غیر نظارتی باشد.
\item بر پایه الگوریتم غیر نظارتی چند‌مرحله‌برای نشانن دادن داده است. اطلاعات مرحله‌های بالاتر از مرحله های پایین تر می‌رسد، تا ساختار سلسله مراتبی را کامل کند.
\end{itemize}
به طور خلاصه دو نوع از نورون های عصبی وجود دارد. 
نوع اول آن ها‌ورودی میگیرد، ونوع دیگری که خروجی می‌دهد. که بین این ها تعداد زیادی لایه های نورونی وجود دارند که خروجی های لایه قبل را پردازش کرده و به لایه بعد می‌دهند. 
تحقیقات‌دراین‌زمینه‌تلاش‌میکنند‌تا‌نحوه‌نمایش‌داده‌های‌بزرگتر را بهتر کنند. بعضی از نمایش ها در این زمینه از علوم اعصاب و نحوه ارتباطی بین سلول های عصبی در مغز انسان الهام میگیرند. 
تکنیک‌های متفاوت یادگیری‌عمیق در زیمنه های رایانه ای مانند تشخیص صدا، تشخیص سخنرانی، پردازش زبان طبیعی به کار گرفته‌می‌شوند. 
\section{خلاصه}
هوش مصنوعی، یادگیری ماشینی و یادگیری عمیق
اساسا ادراک ماشینی هستند. این قدرت تفسیر داده های حسگر ها است. 
 دو روش اصلی که ما چیزها را تفسیر می کنیم، این است که
نامگذاری آنچه حس می کنیم.  به عنوان مثال، صدایی می شنویم و با خودمان می‌گوییم "این صدای دخترم است." 
یا ما یک مه از
فوتون ها  می‌بینیم ومی گوییم "این صورت مادر من است." 
اگر این نام ها نباشند، 
 هنوز هم می توانیم شباهت هاو تنفاوت ها را تشخیص دهیم و
شما ممکن است دو چهره را ببینید و بدانید که آنها
مادر و دختر بودند، بدون اینکه نامشان را بدانید. 
یا
شما ممکن است دو صدا را بشنوید و بدانید که آنها نوعی یکسان دارند. و لهجه آن ها یکسان است.  الگوریتم ها برای نام گذاری از طریق یادگیری نظارتی آموزش می بینند.  و خوشه بندی چیزها
از طریق یادگیری بدون نظارتی آموزش میبینند.  تفاوت میان
یادگیری تحت نظارت و بدون نظارت این است که آیا شما یک
آموزش با برچسب مجموعه ای برای کار با داده ها دارید یا خیر.  برچسب هایی که اعمال می کنید
به داده ها به سادگی نتایجی هستند که شما به آنها اهمیت می دهید. 
شاید شما
به شناسایی افراد در تصاویر اهمیت می دهید.  شاید برایتان مهم باشد که 
شناسایی ایمیل های خشمگین یا هرزنامه هایی که همه فقط حباب های بدون ساختار متن هستند. 
بنابراین یادگیری عمیق،  کار با الگوریتم های دیگر 
 طبقه بندی، خوشه بندی و پیش بینی است.  این کار را با یادگیری 
سیگنال ها، یا ساختار، در داده ها به طور خودکار انجام میدهد.  وقتی الگوریتم های یادگیری عمیق
 آموزش می بینند، آنها در مورد داده ها حدس می زنند،
اندازه گیری خطای حدس های خود در برابر مجموعه آموزشی، و
سپس راه حدس زدن را تصحیح می‌کنند تا به حدس های دقیق تر تبدیل شود. این بهینه سازی است. 
.
حال تصور کنید که با یادگیری عمیق، می توانید طبقه بندی کنید، 
خوشه بندی کنید،  یا پیش بینی کنید هر چیزی که راجب آن دادخ ای دارید،  مانند: تصاویر، ویدیو،
صدا، متن و  سری های زمانی(بازارهای سهام،
جداول اقتصادی و آب و هوا).  یعنی هر چیزی که
انسان ها می توانند حس کنند و فناوری ما می تواند آن را رایانه ای کند.  شما
توانایی تجزیه و تحلیل آنچه در دنیا‌اتفاق می افتد را با یادگیری‌عمیق چند برابر کرده اید.  ما اساساً به جامعه توانایی این را می‌دهیم تا بسیار بیشتر هوشمندانه باشد. 
پیش‌بینی به تنهایی قدرت عظیمی است، 
 طبقه بندی پیش پا افتاده به نظر می رسد، اما با نامگذاری
چیزی، شما می توانید تصمیم بگیرید که چگونه به آن پاسخ دهید. اگر یک ایمیل است، 
هرزنامه، آن را به پوشه هرز ها ارسال می کنید و در زمان خواننده صرفه جویی می کنید.
اگر چهره ای که توسط دوربین درب ورودی شما ثبت شده مادر شماست،
شاید به قفل هوشمند بگویید در را باز کند. اگر اشعه ایکس
یک الگوی تومور را نشان می دهد،  شما آن را برای بررسی عمیق تر علامت گذاری می کنید تا آن را درمان کنید.
در نتیجه کارهایتان را می‌توانید با هوش‌مصنوعی آسان تر کنید.
\cite{ongsulee2017artificial}
\\\
