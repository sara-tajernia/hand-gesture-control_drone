\chapter{نحوه ساخت موسیقی}
% \afterpage{\newpage}
\section{شبکه عصبی}
با داشتن مجموعه داده رمزگذاری شده پنجره ای، داده ها  به شبکه عصبی وارد می شود. نمونه ها و برچسب ها را طوری جدا می کنیم که
در نمونه 
\verb;n;
مثالی با گام های زمانی آن در محدوده 
\verb;,n–1];
\verb;[n–w;
می آوریم و اجازه می دهیم شبکه عصبی . 
نمونه را در
\verb;n + 1;
پیش بینی کند. 
\cref{fig.11}
از پنجره ای از داده های تولید شده از پنجره سازی در 
\cref{fig.9}
استفاده می کند و نشان می دهد که چگونه این
پنجره به شبکه تغذیه می شود. 

\begin{figure}[!h]
\includegraphics[width=10cm,height=10cm]{11.png}
\caption{آموزش شبکه عصبی}\label{fig.11}
\end{figure}

\section{تولید موسیقی}
در فواصل زمانی ثابت در طول آموزش شبکه، شبکه موسیقی تولید می کند و سپس آن را با نوع 
\verb;MIDI;
ذخیره می کند. 
به منظور گوش دادن به آن و تجزیه و تحلیل آن. 
\cref{fig.12}
نشان می دهد که چگونه موسیقی پس از آموزش آن تولید می شود
 . برای تولید موسیقی،  شبکه با یک ماتریس اختلال یا یک ماتریس تصادفی از
مجموعه داده و از آن بخواهید بردار بعدی را پیش بینی کند.  پس از پیش بینی بردار بعدی،  این بردار را به 
ماتریس اضافه می‌کنیم،  و همچنین لیست بردارهای تولید شده و آخرین بردار را از ماتریس خارج می‌کنیم.  به همین ترتیب، همان
فرآیند برای پیش بینی بردار بعدی انجام می شود. این کار چندین بار تکرار می شود تا یک تکه از طول مورد نظر
 به دست می آید. 


\begin{figure}[!h]
\includegraphics[width=10cm,height=10cm]{12.png}
\caption{روش تولید موسیقی}\label{fig.12}
\end{figure}

\section{داده ها و ابزار ها}
\subsection{مجموعه داده ها}

«
کتاب کلاویه خوش خلق و خوی 
\verb;«II;
باخ به عنوان مجموعه داده ما انتخاب شده است . این
به دلیل سبک سیستماتیک موسیقی باخ و همچنین ماهیت خاص این اثر است که باخ در آن
برای هر کلید دو قطعه موسیقی می نویسد. مجموعه داده شامل 24 فایل است که هر فایل شامل دو قطعه است.   
\cref{fig.13}
یک هیستوگرام نرمال شده از تعداد 12 کلید در مجموعه داده را نشان می دهد. هر وقوع یک کلید
بدون توجه به هشتایی شمرده می شود.  این نشان می دهد که همه کلیدها با همان درجه رخ می دهند، یعنی 
هیچ کلید غالب در مجموعه داده وجود ندارد. انحراف استاندارد نمودار میله‌ای عادی شده 0.02683 است. این
نشان می دهد که به طور متوسط ​​تفاوت در تعداد وقوع بین هر دو کلید در مجموعه داده 
حدود 2.6 درصد است.  
\cref{fig.14}
وقوع هر یک از 88 یادداشت را نشان می دهد. ما می توانیم مشاهده کنیم که بیشتر
نت ها در محدوده میانی پیانو قرار دارند که معمولاً در همه موسیقی ها مورد انتظار است. با توجه به این
ویژگی ها، این مجموعه داده به خوبی متعادل است و کاملاً با نیازهای ما مطابقت دارد. 



\begin{figure}[!h]
\includegraphics[width=15cm,height=10cm]{13.png}
\caption{شماره کلید های مجموعه داده ها محور افقی شماره کلید و محور عمودی فرکانس عادی شده}\label{fig.13}
\end{figure}

\begin{figure}[!h]
\includegraphics[width=15cm,height=10cm]{14.png}
\caption{محور افقی شماره کلید پیانو و محور عمودی فرکانش عادی نشده}\label{fig.14}
\end{figure}

\subsection{ابزار های استفاده شده}

ابزارهای مورد استفاده برای اجرای رویکرد پیشنهادی عبارتند از:
\begin{itemize}
\item کراس
\LTRfootnote{keras}
 در باطن تنسورفلو
\LTRfootnote{Tensorflow}
. کراس یک کتابخانه سطح بالا برای یادگیری ماشین با شبکه عصبی است
که به ما اجازه می دهد  تا به سرعت نمونه سازی کنیم و پارامترها را بدون بازنویسی تعداد زیادی کد تغییر دهیم.
تنسورفلو گوگل که کتابخانه ای برای محاسبات روی نمودارها است که کاملاً با شکه های عصبی مطابقت دارد و
به طور موثر بر روی کارت گرافیک اجرا می شود. 
\end{itemize}
\begin{itemize}
\item تخته تنسور
\LTRfootnote{TensorBoard}
. ابزاری از تنسور فلو برای یادگیری تجسمی است.
\item نام‌پای
\LTRfootnote{NumPy}
. نام پای یک کتابخانه علمی جبری خطی است که به طور موثر با ماتریس ها و بردارها سروکار دارد این کتابخانه در کراس استفاده می‌شود و داده‌های آموزشی و آزمایشی باید آرایه‌های نام‌پای قبل از تغذیه آنها به هر شبکه باشد.  علاوه بر این، ما از نام‌پای در بخش رمزگذاری و رمزگشایی برای مقابله با آن استفاده می کنیم. 
ماتریس ها (به عنوان مثال افزودن یک 
 بردار به چند ردیف در یک ماتریس).
\item میدو
\LTRfootnote{Mido}
. میدو یک کتابخانه پایتون است که به خواندن فایل های 
\verb;MIDI; 
و نوشتن آنها کمک می کند. 
\item  دفتر ژوپیتر
\LTRfootnote{Jupyter NoteBook}
. یک برنامه وب برای تجسم داده ها و مدل سازی آماری.  ما از دفترچه ژوپیتر استفاده می کنیم
 تا در آزمایش
بتوانیم بلوک های کد را در هر دنباله ای که انتخاب می کنیم اجرا کنیم. این
به نمونه‌سازی سریع آزمایش‌ها و مدل‌های شبکه‌عصبی کمک می‌کند. 
\end{itemize}
\subsection{شبکه عصبی} 
شبکه‌عصبی در کراس ایجاد شده است و آن را بر روی داده های به دست آمده از فرآیند پنجره سازی آموزش می دهد.  همانطور که در
\cref{fig.15}
نشان داده شده است
، شبکه عصبی از یک یا چند لایه ال.اس.تی.ام(
\verb;LSTM; 
)
تشکیل شده است که هر یک از آنها یک لایه انصرافی به دنبال دارد که کمک می کند.
تمرین را منظم کنید و از اضافه کردن آن جلوگیری کنید. روال انصراف به این صورت است که برخی از گره ها را به صورت تصادفی رها می کند. 
به دنبال آن یک عملیات مسطح انجام می شود که ابعاد ماتریس را از سه بعدی به عنوان خروجی کاهش می دهد.
لایه های 
\verb;LSTM;
را به دو بعدی برای لایه های کاملا متصل تبدیل می کند.  سپس لایه‌های کاملاً متصل را اضافه می‌کنیم و هر کدام را به دنبال آن اضافه می‌کنیم. 
یک لایه ترک تحصیل در نهایت، خروجی شبکه عصبی یک لایه کاملا متصل با اندازه 89 با یک سیگموئید
\LTRfootnote{Sigmoid activation function}
است.
تابع فعال سازی روی آن اعمال می شود. برای راهنمایی آموزش شبکه عصبی، ما از بی نظمی
\LTRfootnote{Entropy}
متقاطع دودویی استفاده کردیم.
تابع ضرر 
\verb;(Log-Loss);
به جای میانگین مربعات خطای کلاسیک. این به دلیل ماهیت ماست
که نزدیک به یک مسئله طبقه بندی چند برچسبی با 88 برچسب و در چند برچسبی است
که طبقه بندی، دودویی متقاطع بی نظمی امتیاز ضرر را بیشتر از دقت مدل معمولی می دهد. 
 به دلیل اینکه بیت 89 یک مقدار واقعی است و یک مقدار دودویی نیست، این تابع از دست دادن
کامل نیست با این حال، برای هدف ما به اندازه کافی نزدیک است. 
\begin{figure}[!h]
\includegraphics[width=12cm,height=17cm]{21.png}
\caption{معماری شبکه عصبی}\label{fig.15}
\end{figure}

برای قضاوت در مورد عملکرد شبکه، داده ها را به 70 درصد داده های آموزشی و 30 درصد اعتبار سنجی تقسیم می کنیم.
. سپس از اندازه گیری اف
\LTRfootnote{F-measure}
که میانگین هارمونیک دقت و یادآوری است برای محاسبه دقت شبکه استفاده می کنیم.
\section{نتایج}
در این فصل نحوه ساخت موسیقی با استفاده از موسیقی که در فصل گذشته به زبان رایانه تبدیل کرده بودیم را دیدیم.
که با استفاده از ابزار های متفاوت چند موسیقی را برای آموزش به یک شبکه عصبی می‌دهیم و موسیقی های جدیدی را تولید میکنیم. و با تابع های متفاوت درستی آن را می سنجیم. اما در اینجا چالش اصلی ما این است که موسیقی به دست آمده برای مخاطب نیز تاثیر گذار باشد. و آیا شبکه عصبی استفاده شده بهترین حالت است یا خیر؟ 
در فصل بعد نتایج به دست آمده از مشاهدات خود را میبینیم و راهی برای از بین بردن این چالش ها ارائه میدهیم. 
\cite{hewahi2019generation}