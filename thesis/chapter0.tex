\chapter{مقدمه}
% \afterpage{\newpage}
\section{مقدمه}
پهپاد‌ها یا به عبارتی هواپیماهای بدون سرنشین امروزه در صنایع مختلف به عنوان یک فناوری
 ‌بسیار گسترده و کارآمد مورد استفاده قرار می‌گیرند. هواپیماهای بدون سرنشین اساساً به عنوان ربات‌های پرنده‌ای
 دیده می‌شوند که عملکردهای متعددی در صنایع مانند جمع‌آوری داده‌ها و سنجش محیط اطراف را بر عهده دارند \cite{waltergesture}.
 از جمله این صنایع می‌توان به کشاورزی، ساخت و ساز، خدمات حمل و نقل و نقشه‌برداری اشاره کرد. یکی از دلایل
 اصلی افزایش کاربرد این هواپیما‌های بدون سرنشین، کارایی بالای آنها است. این فناوری نه تنها به دلیل سرعت بالا در پوشش‌دهی
 مساحت‌های گسترده، بلکه به دلیل قابلیت برنامه‌ریزی و استفاده در صنایع مختلف مورد توجه قرار می‌گیرد.
 همچنین، صرفه‌جویی در هزینه‌های مالی و جانی و افزایش امنیت نیز از جمله عوامل مهمی است که اهمیت پهپادها را بیشتر می‌کند\cite{puri2017agriculture}.
 \\
 در حال حاضر، ربات های پرنده در مشاغلی همچون سیستم‌های تحویل بسته استفاده می‌شوند \cite{gatteschi2015new}. به عنوان مثال، شرکت‌هایی مانند آمازون و 
 \lr{UPS} از پهپادهابرای تحویل بسته‌های خود استفاده می‌کنند \cite{moore2014nypd}. 
 در پی این موضوع، بسیاری از شرکت‌های تولید کننده
  پهپاد تشویق شده‌اند تا انواع مختلفی از ویژگی‌های نرم‌افزاری و سخت‌افزاری
 مانند حسگرها را به پهپاد‌ها اضافه کنند، که ابتدایی ترین
  آنها دوربین است. دوربین بصری یک حسگر ضروری برای پهپادهای فعلی
  است که آنها را به پهپادهای کاربردی و متعدد در بازار تبدیل می‌کند\cite{natarajan2018hand}. همراه با این تغییرات زمینه مطالعاتی جدیدی به نام رابط هواپیماهای بدون سرنشین و
   انسان گشوده‌شد تا تعامل بین پهپاد‌ها و 
 انسان را پیشرفت دهد. این تعامل با استفاده از مجموعه‌ دستگاه‌های سنتی مانند کنترلر‌های رادیویی\LTRfootnote{Radio Controller} و یا کنترل پهپادها با استفاده از وضعیت بدن و دست انسان را انجام می‌شود \cite{hadri2018hand}.
 \\
 یکی از رویکردهای مورد استفاده برای افزایش کاربرد و دسترسی به پهپادها، استفاده از روش‌های مبتنی بر بینایی ماشین است. این کار معمولا از طریق پردازش تصویر
 همراه با استفاده از شبکه‌های عصبی
 \LTRfootnote{Neural Netwroks}
 انجام می‌شود. پهپاد‌هایی که با مدل‌های بینایی ماشین آموزش می‌بینند، توانایی تحلیل تصاویر و ویدئو‌هایی که از محیط اطراف
 دریافت می‌کنند را دارا هستند. این قابلیت به پهپاد این امکان را می‌دهد که بدون نیاز به تداخل انسانی، وظایفی همچون امنیت، ارسال کالا، تصویر‌برداری و ... را انجام دهد\cite{zhu2018vision}.
 می‌توان گفت که هدف اصلی استفاده از بینایی ماشین در پهپاد‌ها به حداقل رساندن دخالت انسان به صورت مستقیم است. این
 امر پهپاد را قادر می‌سازد تا تشخیص اشیاء، تشخیص چهره، تحلیل تصاویر، شناسایی الگوهای مختلف و مواردی از این دست را به صورت خودکار انجام دهند \cite{guvenc2018detection}.

 \section{چالش‌های استفاده از پهپاد}
 استفاده از پهپادها، با چالش‌های متعددی همراه است. یکی از این چالش‌ها، محدودیت زمان پرواز است که پس از مدت کوتاهی پهپاد‌ها نیاز به شارژ مجدد دارند. 
 همچنین، محدودیت‌های محیطی نیز می‌توانند به چالش‌های سختی تبدیل شوند؛ زیرا پهپادها به شرایط محیطی مانند آب و هوا و ارتفاع حساس هستند و این موارد می‌تواند 
 در عملکرد آنها تأثیر به‌سزایی داشته باشد. در ادامه باید به میزان اهمیت امنیت اطلاعات به دست آمده از پهپاد‌ها نیز اشاره کرد، پهپادها به دلیل استفاده از سیستم‌های موقعیت‌یاب و ارتباطات بی‌سیم ممکن 
 است در برابر حملات سایبری آسیب‌پذیر باشند و این آسیب پذیری‌ها اطلاعات مهمی را که توسط آنها مخابره می‌شود در معرض خطر قرار می‌دهد.
 \\
 همچنین می‌توان به چالش‌هایی که ما در این پروژه با آنها سر و کار داریم و در تلاش می‌کنیم تا آنها را از بین ببریم و یا کمتر کنیم نیز اشاره کرد. 
   یکی از این چالش‌ها انتقال اطلاعات است.برای ارتباط با پهپادها از شبکه‌های بی‌سیم استفاده می‌شود و در شرایطی مانند اشباع شبکه  و یا افزایش فاصله بین پهپاد و کنترل‌کننده، ممکن است این ارتباط دچار اختلال شود.
 علاوه بر این، محدودیت محاسباتی پهپاد نیز با توجه به اهدافی که برای آن در نظر گرفته شده می‌تواند چالش برانگیز باشد؛ زیرا پهپادها به دلیل محدودیت‌های سخت‌افزاری و نرم‌افزاری، دارای پردازشگرها و حافظه‌های محدودی هستند \cite{hassanalian2017classifications}.
 قابل ذکر است که با ادامه پیشرفت فناوری پهپاد، می‌توان انتظار داشت که ویژگی‌های جدید و نوآورانه‌ای برای از بین بردن این محدودیت‌ها و چالش‌ها به‌ پهپادهای آینده اضافه شود.

 \section{اهمیت استفاده از بینایی ماشین در پهپاد}
 طبق اعلام پیش‌بینی اداره هوانوردی فدرال
 \LTRfootnote{Federal Aviation Administration}
 ، بازار هواپیماهای بدون سرنشین تا سال 2025 به 17 میلیارد دلار خواهد رسید و 7 میلیون هواپیمای بدون سرنشین به آسمان پرواز خواهند‌ کرد. پهپادهای کنترل
 از راه دور به تدریج به دستگاه‌های نیمه خودکار یا کاملاً خودکار تبدیل می‌شوند که از این دستگاه‌‌ها از پیاده‌سازی‌های مبتنی بر هوش مصنوعی بهره می‌برند. 
 در این پروژه هدف ما هدایت پهپاد با استفاده از علائم دست مبتنی بر بینایی ماشین است که یک حوزه پژوهشی مهم در ترکیب هوش مصنوعی و رباتیک است. 
 استفاده از حرکات دست در کنترل هواپیماهای بدون سرنشین در حال تبدیل شدن به یک روش محبوب برای تعامل بین کاربر و پهپاد است. 
 

\section{تعریف مسئله}

این پایان نامه یک سیستم کامل برای کنترل هواپیماهای بدون سرنشین 
با استفاده از حرکات دست پیشنهاد می‌کند. سیستم پیشنهادی باید به صورت بی درنگ  \LTRfootnote{Real-Time} کار کند و دقت\LTRfootnote{Accuracy} بالایی را داشته باشد تا بتواند به بهترین نحو ممکن پهپاد را کنترل کند \cite{hadri2018hand}.
\\
در این روش، از سیستم بینایی ماشین به منظور تشخیص و تحلیل حرکات دست از روی تصاویر ویدئویی پهپاد استفاده می‌شود. با استفاده از الگوریتم‌های یادگیری عمیق و شبکه‌های عصبی، سیستم 
قادر است علائم و حرکات دست را تشخیص داده و به تفسیر آنها بپردازد. سپس، براساس تحلیل این حرکات، دستورات مربوطه برای حرکت و کنترل پهپاد را صادر کند.
 این روش نه تنها از دقت بالا برای تشخیص و تفسیر حرکات دست برخوردار است، بلکه قابلیت ارائه یک رابط کاربری بین انسان و پهپاد را نیز فراهم می‌کند. 
به طوری که با استفاده از حرکات دست کاربر قادر است به راحتی و بدون نیاز به دستگاه‌های کنترل خارجی، پهپاد را هدایت کند \cite{yoo2022motion}.
\\
استفاده از حرکات دست برای کنترل پهپاد مزایای زیادی دارد. ابتدا باید گفت که حرکات دست یک شکل طبیعی ارتباطی هستند و استفاده از آنها برای کنترل پهپاد یک روش شهودی و طبیعی برای تعامل با فناوری است
. این امر باعث می‌شود که کاربران بتوانند به راحتی و با کمترین تلاش پهپاد را کنترل کنند. استفاده از حرکات دست به کاربر اجازه می‌دهد پهپاد را با سرعت و دقت
بیشتری کنترل کند و محدودیت‌های مرتبط با دستگاه‌های کنترل سنتی را کاهش دهد. همچنین، این روش حرکت و دنبال کردن پهپاد را آسان‌تر می‌کند و امکان جابجایی پهپاد در فضا‌های باز را بهبود می‌بخشد.
\\
استفاده از علائم دست سبب کاهش نیاز به دستگاه‌های کنترلی پیچیده می‌شود و به این ترتیب، پهپاد را برای طیف وسیع‌تری از کاربران قابل دسترس می‌کند.
این امر به کاربرانی که با دستگاه‌های کنترل سنتی آشنایی ندارند، امکان استفاده آسان از پهپاد را می‌دهد. همچنین، با توجه به چالش‌هایی که در بخش قبلی بیان شده است، 
این روش خطرات مرتبط با اتصالات بی‌سیم بین کنترلر و پهپاد را کاهش می‌دهد و دقت کنترل پهپاد در محیط‌های پرتلاطم و متفاوت از نظر آب و هوایی را افزایش می‌دهد. 


برای پیاده‌سازی این پروژه از شبکه‌های عصبی عمیق \LTRfootnote{Deep Neural Network}
، مانند شبکه‌های عصبی پیچشی، استفاده شده است. دلیل استفاده از این معماری‌ها قابلیت استخراج خودکار ویژگی‌ها با توجه به الگوریتم دسته‌بندی تصاویر\LTRfootnote{Image Classification}
است. عملکرد شبکه‌های عصبی پیچشی به این گونه است که ویژگی‌ها را با استفاده از  لایه‌های پنهان
\LTRfootnote{Hidden Layers}
می‌آموزد، همچنین می‌تواند تعداد پارامترها را بدون به خطر انداختن دقت مدل تغییر دهد.

\subsection{چالش‌های اجرای پروژه}
وجود سخت‌افزاری مناسب برای اجرای این پروژه الزامی است. پهپاد انتخاب شده در ابتدا باید شامل یک دوربین با کیفیت تصویر
\LTRfootnote{Resolution}
نسبتا بالا باشد تا بتوان علائم دست تا فاصله سه متری از پهپاد به وضوح تشخيص داده شود.
در ادامه از آنجایی که بی‌درنگ بودن در این پروژه از اهمیت بالایی برخوردار است پهپاد باید پردازنده نسبتا قوی داشته باشد تا بتواند در کمترین زمان ممکن ویدیو را از دوربین دریافت کرده و انتقال دهد و پس از به دست آوردن خروجی سیستم دستور متناسب را اجرا کند. از دیدگاهی دیگر، از آنجایی که این ارتباطات در کنار حرکت پهپاد انرژی زیادی می‌طلبد، لذا باید پهپادی را انتخاب کرد
که از نظر باطری بادوام و باکیفیت باشد تا به مرور زمان برای استفاده کننده آزاردهنده نباشد.
\\
همچنین پهپاد مد نظر ما برای این پروژه باید توانایی ارتباط با  زبان برنامه‌نویسی پایتون را نیز داشته باشد تا بتوان دستورهای پیش‌بینی شده توسط شبکه‌های هوش مصنوعی پیاده‌سازی شده با این زبان را روی آن اجرا کرد.


\section{مراحل انجام پروژه}
\begin{enumerate}
    \item  انتخاب علائم‌ مناسب و مفهومی برای کنترل پهپاد
    \item  پیاده‌سازی کد برای جمع‌آوری مجموعه‌داده
    \item  پیاده‌سازی شبکه مربوط به پیدا کردن کف دست 
    \item  پیاده‌سازی شبکه مرتبط با پیدا کردن نقاط کلیدی دست 
    \item  پیاده‌سازی شبکه‌هایی برای تعیین علائم دست
    \item  آموزش شبکه‌ها
    \item  بهینه‌سازی شبکه‌ها
    \item  تست مدل‌ها و انتخاب بهترین مدل
    \item  پیاده سازی رأی‌گیری پنجره‌ای\LTRfootnote{Window Voting}
    \item  اجرای مدل روی پهپاد
\end{enumerate}
 

\section{جمع‌بندی}
 هدف این پروژه پیاده‌سازی برنامه‌ای کاربردی بر روی پهپاد است تا بتوان نه علائم دست از پیش تعیین شده را شناسایی و با توجه به آنها دستور مرتبط با هر یک را به پهپاد ارسال کند..
 در این پروژه موارد زیر از اهمیت بالایی برخوردار هستند:
 \begin{itemize}
    \item دقت بالای شبکه: در صورت انجام نادرست دستورات امکان از دست رفتن مقدار هنگفتی هزینه مالی شامل آسیب‌های وارد به پهپاد و نیروی انسانی وجود دارد.
    \item پیچیدگی کم و سرعت بالای تشخیص علائم دست: بی‌درنگ بودن اجرای دستورات مهم است و باید در کمترین زمان ممکن رخ دهد تا مورد پسند کاربر باشد.
\end{itemize}

در فصل بعدی به بررسی کار‌های مشابهی که در این زمینه وجود دارند خواهیم پرداخت تا بتوانیم با استفاده از آن‌ها دقت سیستم خود را مقایسه کرده و از نظر 
عملکردی نتیجه قابل قبولی را به دست آوریم.


