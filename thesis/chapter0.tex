\chapter{مقدمه}
% \afterpage{\newpage}
\section{مقدمه}
پهپاد‌ها یا به عبارتی هواپیماهای بدون سرنشین امروزه در صنایع مختلف به عنوان یک فناوری
 ‌بسیار گسترده و کارآمد مورد استفاده قرار می‌گیرند. هواپیماهای بدون سرنشین اساساً به عنوان ربات‌های پرنده‌ای
 دیده می‌شوند که عملکردهای متعددی مانند جمع آوری داده‌ها و سنجش از محیط اطراف را بر عهده دارند \cite{waltergesture}.
 از جمله این صنایع می‌توان به کشاورزی، ساخت و ساز، خدمات حمل و نقل و نقشه‌برداری اشاره کرد. یکی از دلایل
 اصلی افزایش کاربرد این هواپیما‌های بدون سرنشین، کارایی بالای آنها است. این فناوری نه تنها به دلیل سرعت بالا در پوشش‌دهی
 مساحت‌های گسترده، بلکه به دلیل قابلیت برنامه‌ریزی و استفاده در صنایع مختلف مورد توجه قرار می‌گیرد.
 همچنین، صرفه‌جویی در هزینه‌های مالی و جانی و افزایش امنیت نیز از جمله عوامل مهمی است که اهمیت پهپادها را بیشتر می‌کند\cite{puri2017agriculture}.
 \\
 در حال حاضر، ربات های پرنده در مشاغل مختلف مانند سیستم های تحویل بسته استفاده می‌شوند \cite{gatteschi2015new}. به عنوان مثال، شرکت‌هایی مانند آمازون و 
 \lr{UPS} از پهپادهای چند روتور برای تحویل بسته‌های خود استفاده می‌کنند \cite{moore2014nypd}. 
 در پی این موضوع، بسیاری از شرکت های تولید کننده
  پهپاد تشویق شدند تا انواع مختلفی از ویژگی‌های نرم‌افزاری و سخت‌افزاری
 مانند حسگرها را به پهپاد‌ها اضافه کنند، که ابتدایی ترین
  آنها دوربین است. دوربین بصری یک حسگر ضروری برای پهپادهای فعلی
  است. هزینه کم، قدرت کم، اندازه کوچک ضبط تصویر، و دستگاه های پخش جریان، آنها را به پهپادهای کاربردی و متعدد در بازار تبدیل می‌کند\cite{natarajan2018hand}. در ادامه زمینه مطالعاتی جدیدی به نام رابط هواپیماهای بدون سرنشین و
   انسان\LTRfootnote{Human drone interface} گشوده شد تا تعامل بین پهپاد و 
 انسان را پیشرفت دهد، که این تعامل مجموعه‌ دستگاه‌های سنتی مانند کنترلر رادیویی\LTRfootnote{Radio Controller} تا کنترل پهپادها با استفاده از وضعیت بدن و دست انسان را شامل می‌شود \cite{hadri2018hand}.
 \\
 یکی از رویکردهای مورد استفاده برای افزایش کاربرد و دسترسی به پهپادها، استفاده از بینایی ماشین است. این ویژگی معمولا از طریق پردازش تصویر
 و با استفاده از شبکه‌های عصبی به کار می‌رود. پهپاد‌هایی که با مدل‌های بینایی ماشین آموزش می‌بینند، توانایی تحلیل تصاویر و ویدئو‌هایی که از محیط اطراف
 دریافت می‌کنند را دارا هستند. این قابلیت به پهپاد این امکان را می‌دهد که بدون نیاز به تداخل انسانی، وظایفی همچون امنیت، ارسال کالا، پست و این چنین موارد را انجام دهد\cite{zhu2018vision}.
 می‌توان گفت هدف اصلی استفاده از بینایی ماشین در پهپاد‌ها برای به حداقل رساندن دخالت انسان به صورت مستقیم است. این
 امر پهپاد را قادر می‌سازد تا تشخیص اشیاء، تشخیص چهره، تحلیل تصاویر، شناسایی الگوهای مختلف و مواردی از این دست را به صورت خودکار انجام دهند \cite{guvenc2018detection}.

 \section{چالش‌های استفاده از پهپاد}
 استفاده از پهپادها، با چالش‌های متعددی همراه است. یکی از این چالش‌ها، محدودیت زمان پرواز است که پس از مدتی نیاز به شارژ مجدد دارند. 
 همچنین، محدودیت‌های محیطی نیز می‌تواند به چالش‌هایی بدل شوند؛ زیرا پهپادها به شرایط محیطی مانند آب و هوا، یا وزن و ارتفاع حساس هستند و این موارد می‌تواند 
 در طراحی آنها تأثیر به‌سزایی داشته باشد. در ادامه باید به میزان اهمیت امنیت اطلاعات هم اشاره کرد، زیرا پهپادها به دلیل استفاده از سیستم‌های موقعیت‌یاب و ارتباطات بی‌سیم ممکن 
 است در برابر حملات سایبری آسیب‌پذیر باشند و اطلاعات مهمی که توسط آنها مخابره می‌شود، در معرض خطر قرار گیرد.
 \\
 همچنین می‌توان به برخی چالش‌هایی که ما هم در این پروژه به صورتی با آنها سر و کار داریم و در تلاشیم آنها را از بین ببریم یا کمتر کنیم اشاره کرد. 
 مانند انتقال اطلاعات، زیرا برای ارتباط با پهپادها از شبکه‌های بی‌سیم استفاده می‌شود و در شرایطی مانند اشباع شبکه یا فاصله بین پهپاد و کنترل‌کننده، ممکن است این ارتباط دچار مشکل شود.
 علاوه بر این، محدودیت محاسباتی پهپاد نیز با توجه به اهدافی که برای آن در نظر گرفته شده می‌تواند چالش برانگیز باشد؛ زیرا پهپادها به دلیل محدودیت‌های سخت‌افزاری و نرم‌افزاری، دارای پردازشگرها و حافظه‌های محدودی هستند \cite{hassanalian2017classifications}.
 قابل ذکر است که با ادامه پیشرفت فناوری پهپاد، می‌توان انتظار داشت که ویژگی‌های جدید و نوآورانه‌ای برای از بین بردن این محدودیت‌ها و چالش‌ها به‌ پهپادهای آینده اضافه شود.

 \section{اهمیت استفاده از بینایی ماشین در پهپاد}
 طبق اعلام پیش‌بینی اداره هوانوردی فدرال، بازار هواپیماهای بدون سرنشین تا سال 2025 به 17 میلیارد خواهد رسید و 7 میلیون هواپیمای بدون سرنشین به آسمان پرواز خواهند‌ کرد. پهپادهای کنترل
 از راه دور به تدریج به دستگاه های نیمه خودکار یا کاملاً خودکار تبدیل می‌شوند که از پیاده سازی مبتنی بر هوش مصنوعی بهره می‌برند. 
 در این پروژه هدف ما هدایت پهپاد با استفاده از علائم دست مبتنی بر بینایی ماشین است که یک حوزه پژوهشی مهم در ترکیب هوش مصنوعی و رباتیک است. 
 استفاده از حرکات دست در کنترل هواپیماهای بدون سرنشین در حال تبدیل شدن به یک روش محبوب برای تعامل است. این پایان نامه یک سیستم کامل برای کنترل هواپیماهای بدون سرنشین 
 با استفاده از حرکات دست پیشنهاد می‌کند. سیستم پیشنهادی باید در زمان واقعی\LTRfootnote{Real-time} کار کند و دقت\LTRfootnote{Accuracy}خوبی داشته باشد تا بتواند به بهترین نحو ممکن پهپاد را کنترل کند \cite{hadri2018hand}.
 \\
 در این روش، از سیستم بینایی ماشین به منظور تشخیص و تحلیل حرکات دست از روی تصاویر ویدئویی پهپاد استفاده می‌شود. با استفاده از الگوریتم‌های یادگیری عمیق و شبکه‌های عصبی، سیستم 
 قادر است علائم و حرکات دست را تشخیص داده و به تفسیر آنها بپردازد. سپس، براساس تحلیل این حرکات، دستورات مربوطه برای حرکت و کنترل پهپاد را صادر کند.
 بدین صورت این روش نه تنها از دقت بالا برای تشخیص و تفسیر حرکات دست برخوردار است، بلکه قابلیت ارائه یک رابط کاربری بین انسان و پهپاد را نیز فراهم می‌کند. 
 به طوری که با استفاده از حرکات دست کاربر قادر است به راحتی و بدون نیاز به دستگاه‌های کنترل خارجی، پهپاد را هدایت کند \cite{yoo2022motion}.
\\
استفاده از حرکات دست برای کنترل پهپاد مزایای زیادی دارد. ابتدا باید گفت که حرکات دست یک شکل طبیعی ارتباطی هستند و استفاده از آنها برای کنترل پهپاد یک روش شهودی و طبیعی برای تعامل با فناوری است.
. این امر باعث می‌شود که کاربران بتوانند به راحتی و با کمترین تلاش پهپاد را کنترل کنند. استفاده از حرکات دست به کاربر اجازه می‌دهد پهپاد را با سرعت و دقت
بیشتری کنترل کند و محدودیت‌های مرتبط با دستگاه‌های کنترل سنتی را کاهش دهد. همچنین، این روش، حرکت و دنبال کردن پهپاد را آسان‌تر می‌کند و امکان جابجایی پهپاد در فضا را بهبود می‌بخشد.
\\
استفاده از علائم دست سبب کاهش نیاز به دستگاه‌های کنترل پیچیده می‌شود و به این ترتیب، پهپاد را برای طیف وسیع‌تری از کاربران قابل دسترس می‌کند.
این امر به کاربرانی که با دستگاه‌های کنترل سنتی آشنایی ندارند، امکان استفاده آسان از پهپاد را می‌دهد. همچنین، با توجه به چالش‌هایی که از قبل بیان شده است، 
این روش خطرات مرتبط با اتصالات بی‌سیم بین کنترلر و پهپاد را کاهش می‌دهد و دقت در کنترل پهپاد در محیط‌های پرتلاطم و مختلف را افزایش می‌دهد. 

\section{تعریف مسئله}
هدف این پروژه کنترل‌کردن پهپاد با استفاده از پردازش تصویر\LTRfootnote{Image Processing} در زمان واقعی است. برای پیاده‌سازی آن می‌توان از یک شبکه عصبی عمیق \LTRfootnote{Deep Neural Network}
، مانند یک شبکه عصبی کانولوشن \LTRfootnote{Convolutional Neural Network(CNN)} ، استفاده کرد. دلیل استفاده از این معماری قابلیت استخراج خودکار ویژگی‌ها با توجه به الگوریتم طبقه‌بندی تصاویر\LTRfootnote{Image Classification}
است. عملکرد شبکه عصبی کانولوشنال به این گونه است که ویژگی‌ها را با توجه به لایه‌های پنهان می‌آموزد، همچنین می‌تواند تعداد پارامترها را بدون به خطر انداختن دقت مدل تغییر دهد. با گذشت زمان محققان 
معماری‌های مختلفی از شبکه عصبی کانولوشن را برای دقت\LTRfootnote{Accuracy} بهتر، زمان پردازش کمتر و پیچیدگی های \LTRfootnote{Complexity} گوناگون مطرح کردند. 

\subsection{چالش‌های اجرای پروژه}
وجود سخت‌افزاری مناسب برای اجرای این پروژه الزامی است. پهپاد انتخاب شده در ابتدا باید شامل یک دوربین با رزولوشن نسبتا بالا (حداقل **** پیکسل باشد) تا  ژست دست تا فاصله سه متری از پهپاد به وضوح گرفته شود.
در ادامه از آنجایی که زمان واقعی در این پروژه از اهمیت بالایی برخوردار است پهپاد باید پردازنده نسبتا قوی داشته باشد تا بتواند به صورت مستقل و بدون نیاز به هیچ‌گونه سخت افزار خارجی مدل را اجرا کند، 
بدین صورت که در هر لحظه ورودی عکس گرفته‌شده از دوربین را به مدل بدهد و در کمترین زمان ممکن بتواند خروجی مدل را به دست آورده و دستور مورد نظر را روی 
پهپاد به اجرا درآورد. از دیدگاهی دیگر، از آنجایی که اجرای یک مدل بینایی ماشین یک برنامه سنگین است و اجرای آن برای عموم پهپاد‌ها انرژی زیادی میطلبد، لذا باید پهپادی را انتخاب کرد
که از شامل باطری بادوام و باکیفیت باشد که هم در هنگام اجرای مدل بتواند انرژی مورد‌نیاز پردازنده را فراهم کند و همچنین عمر کوتاه آن به مرور زمان برای استفاده کننده آزاردهنده نباشد.



\section{مراحل انجام پروژه}
\begin{enumerate}
    \item  انتخاب ژست‌های مناسب و مفهومی برای کنترل پهپاد
    \item  انتخاب چگونگی دخیره کردن دیتای عکس‌ها در دیتاست
    \item  پیاده‌سازی کد برای جمع‌آوری دیتاست
    \item  پیاده‌سازی مدل‌ها متناسب با دیتا
    \item  آموزش به مدل‌ها
    \item  بهینه‌سازی مدل‌ها
    \item  تست مدل‌ها و انتخاب بهترین مدل
    \item  پیاده سازی رأی‌گیری پنجره‌ای\LTRfootnote{Window voting}
    \item  اجرای مدل روی پهپاد
\end{enumerate}
 

\section{جمع بندی}
 هدف این پروژه پیاده‌سازی برنامه‌ای کاربردی بر روی پهپاد است تا بتواند 9 ژست دست از پیش تعیین شده را شناسایی و با توجه به آنها دستور داده شده از طرف کاربر را به پهپاد بدهد.
 در این پروژه موارد زیر از اهمیت بالایی برخوردار هستند:
 \begin{itemize}
    \item دقت بالای مدل زیرا در صورت انجام نادرست دستورات امکان برآورد هزینه مالی وجود دارد.
    \item زمان واقعی زیرا اجرای دستورات باید در کمترین زمان ممکن رخ دهد تا مورد پسند کاربر باشد.
    \item مدلی سبک تا توانایی اجرا روی پردازنده پهپاد را داشته‌باشد.
\end{itemize}


