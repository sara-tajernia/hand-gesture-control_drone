\chapter{نتایج و ارزیابی}
\section{مقدمه}
سیستم تشخیص حرکات دست به نقش مهمی در ایجاد تعامل کارآمد بین انسان و ماشین تبدیل شده است. پیاده‌سازی این سیستم با استفاده از تشخیص ژست دست، نوید گستره وسیعی از کاربردها را در صنعت فناوری می‌دهد. در این پروژه، 
معماری‌های گوناگونی مانند شبکه‌های عصبی پیچشی، شبکه‌های حافظه کوتاه‌مدت بلند، و شبکه‌های عصبی چندلایه مورد آزمایش قرار گرفتند تا بهترین پیاده‌سازی برای تشخیص حرکات دست انتخاب شود. 
\\
در این فصل، به بررسی نتایج به‌دست‌آمده از این پروژه پیاده‌سازی شده می‌پردازیم و دقت و عملکرد سیستم در شرایط مختلف را ارزیابی می‌کنیم. همچنین، نقاط قوت و ضعف هر یک از معماری‌های مورد استفاده 
را تحلیل کرده و پیشنهاداتی برای بهبود سیستم ارائه خواهیم داد. هدف این فصل، ارائه یک تحلیل جامع از کارایی سیستم و شناخت دقیق‌تر از عواملی است که می‌توانند به ارتقاء عملکرد آن کمک کنند.


\section{ارزیابی دقت سیستم}

\section{دقت تشخیص ژست‌ها}
تحلیل دقت مدل‌ها در شرایط مختلف
مقایسه دقت بین معماری‌های مختلف (CNN، LSTM، MLP)

\section{نرخ خطا}
بررسی و تحلیل نرخ خطاها
دلایل خطا و راهکارهای پیشنهادی برای کاهش آنها


\section{عملکرد زمان‌بندی}


\section{سرعت اجرای برنامه}
زمان پاسخگویی سیستم
تاثیر استفاده از کارت گرافیکی بر سرعت

\section{سخت‌افزار مورد نیاز}
این پروژه باید به گونه‌ای اجرا می‌شد که بر روی ساده‌ترین سیستم‌های کامپیوتری نیز قابل اجرا باشد، زیرا سخت‌افزار پهپادها به‌طور معمول دارای پردازنده‌های ضعیف‌تری هستند. همچنین، استفاده از کارت گرافیکی ممکن نبود، چرا که 
پهپادها فاقد کارت گرافیکی می‌باشند. معماری‌های پیاده‌سازی شده به نحوی طراحی شدند که تعادل میان دقت و بهره‌وری از سخت‌افزار حفظ شود، به طوری که هم قابلیت اجرای زمان واقعی داشته باشند و هم امکان پیاده‌سازی آن‌ها بر روی پهپاد 
فراهم باشد. از این رو، معماری‌ها به گونه‌ای پیاده‌سازی شدند که بر روی پردازنده اجرا شوند و میزان استفاده از پردازنده برای آن‌ها به شرح زیر است:


\begin{table}[h!]
    \centering
    \begin{tabular}{||c c c||}
     \hline
     \rule{0pt}{3.5ex}معماری مدل & پردازنده & فضای ذخیره‌شده \\ [1.5ex]
     \hline
     \rule{0pt}{2.5ex} & & \\  % Adds space before the first data row, while keeping the vertical lines
     \lr{MLP} & 6 & 87837 \\ [2.5ex]
     \lr{CNN} & 7 & 78 \\ [2.5ex]
     \lr{LSTM} & 545 & 778 \\ [2.5ex]
     \hline
    \end{tabular}
    \caption{جدول}
    \label{table:1}
\end{table}



\section{نتایج و ارزیابی}

این پروژه با مدل‌های گوناگونی پیاده‌سازی شد تا بتوان بهترین آنها را برای نتیجه نهایی بر روی پهپاد اجرا کرد. نمونه خروجی این مدل‌ها به شرح زیر است.

\begin{table}[h!]
    \centering
    \begin{tabular}{||c c c c||} 
     \hline
     معماری مدل & دوره\LTRfootnote{Epoch} & دقت آموزش & دقت تست \\ [0.5ex] 
     \hline\hline
     \lr{MLP} & 6 & 87837 & 787 \\ 
     \lr{CNN} & 7 & 78 & 5415 \\
     \lr{LSTM} & 545 & 778 & 7507 \\
     4 & 545 & 18744 & 7560 \\
     5 & 88 & 788 & 6344 \\ [1ex] 
     \hline
    \end{tabular}
    \caption{چدول}
    \label{table:2i}
\end{table}

\section{جمع‌بندی}
