\chapter{نحوه تبدیل موسیقی به زبان رایانه}
\thispagestyle{empty}
% \afterpage{\newpage}

\section{مقاطلاعات پس زمینه موسیقی}\label{ music}
درک اینکه چگونه مولد موسیقی ما کار می کند نیاز به دانش پیش زمینه در برخی از مفاهیم اساسی دارد. 
تئوری موسیقی. این بخش شامل خلاصه ای بسیار کوتاه از اصطلاحات ضروری موسیقی است. اصطلاح گام صدا 
برای توصیف بلندی یا پایین بودن صدا استفاده می شود.  در موسیقی هفت نت (یا "صدا") وجود دارد که 
\verb;A,;
\verb;B,; 
\verb;C,; 
\verb;D,; 
\verb;E,; 
\verb;F,; 
\verb;G;
. هر مجموعه نت
(
\verb;A;
تا 
\verb;G;
) متعلق به یک هشتایی
\LTRfootnote{Octave
}
هستند. از هشتایی برای تعیین این استفاده می‌شود که تعیین کنیم که نت ها در کدام گام اند.  
. برای مثال، 
\verb;A1;
(
\verb;A;
در هشتایی 1)
بسیار عمیق تر و پایین تر از
\verb;A7;
(
\verb;A;
در 
\verb;A;
) 
خواهد بود.
هشتایی 7) که گام بالاتر و صدای تندتر است.  یک کیبورد معمولی پیانو شامل
88 کلید قابل پخش و هفت هشتایی کامل همانطور که با رنگ های 
\cref{fig.1}
نشان داده شده است.
\begin{figure}[!h]
\includegraphics[height=4cm]{1.jpg}
\caption{کیبورد پیانو}\label{fig.1}
\end{figure}
اکثر قطعات موسیقی در یک کلید هستند. یک کلید، یادداشتی است که پایه (یا "خانه") را تشکیل می دهد
\verb;A;
موسیقی در قطعه یک قطعه نواخته شده در کلید
\verb;E;
به این معنی است که نت
\verb;E;
موسیقی در آهنگ است .
ترازو مجموعه خاصی از یادداشت ها است که بر پایه کلید ساخته شده اند. این را می توان به عنوان یک مجموعه ابزار خاص توصیف کرد
یادداشت هایی که با آن می توان آهنگ ایجاد کرد . ترازوها با یکدیگر متفاوت هستند
از این نظر که برخی برای موسیقی "شادتر" یا "روشن تر" (
در مقیاس 
\verb;C;
ماژور) و برخی برای "غمگین" بهتر هستند.
یا قطعات "تاریک" مانند مقیاس های کوچک 
\verb;D;
یا
\verb;E;
. آکورد گروهی از نت‌ها است که به طور همزمان نواخته می‌شوند و صدایی متمایز را به‌صورت گروهی تولید می‌کنند. مثلا،
آکورد 
\verb;A Major; 
از نت های 
\verb;A;
،
\verb;C; 
و 
\verb;E; 
تشکیل شده است.  در موسیقی، جابجایی 
فرآیند "تغییر یک ملودی است،  یک پیشرفت هارمونیک یا یک قطعه موسیقی کامل به یک کلید دیگر"
در حالی که همان ساختار لحن را حفظ می کند». این به این معنی است که ساختار آهنگ 
در حین تغییر کلید حفظ می شود.
\subsection{نوع داده
}

فرمت رابط دیجیتال ابزار موسیقی میدی (
\verb;MIDI;
یا 
\verb;.mid;
) 
برای ذخیره دستورالعمل های پیام استفاده می شود.  شامل نت، فشار/حجم، سرعت، زمان شروع و پایان آنها، عبارات و غیره است. 
صدا را مانند فرمت‌های صوتی ذخیره نمی‌کند،  بلکه دستورالعمل‌هایی را در مورد نحوه تولید صدا ذخیره می‌کند. 
این دستورالعمل ها را می توان توسط یک کارت صوتی که از یک جدول موج (جدول صدای ضبط شده) استفاده می کند، تفسیر کرد.
امواج) برای ترجمه پیام های 
\verb;MIDI;
به صدای واقعی.این ها
توسط نرم‌افزار استودیو پخش‌کننده 
\verb;midi;
مانند 
\verb;Fruity Loops (FL) Studio;
یا ترتیب‌دهنده‌های محبوب مانند سنتزیا (
\verb;Synthesia;)
(نگاه کنید به
\cref{fig.2}) نیز قابل تفسیر هستند.  نرم افزارهای نت نویسی موسیقی مانند موس اسکور( 
\verb;MuseScore;)
یا فاینال ( 
\verb;Finale;)
می توانند 
\verb;midi; 
را به نت های قابل ویرایش ترجمه کنند. 
این به کاربران این امکان را می دهد که موسیقی را با نماد موسیقی معمولی روی رایانه خود بنویسند و سپس با 
\verb;MIDI; 
به آن گوش دهند
مانند یک نوازنده بدون نیاز به ساز .
\begin{figure}[!h]
\includegraphics[height=4cm]{2.png}
\caption{نمونه یک تصویر از صفحه برنامه 
}\label{fig.2}
\end{figure}
پیام های 
\verb;MIDI;
به پیام های سیستم و کانال جدا می شوند. از آنجایی که ما در درجه اول نگران هستیم
در یادداشت ها، ما فقط روی پیام های کانال تمرکز می کنیم. 16 کانال 
\verb;MIDI;
ممکن (16 جریان از
پیام های کانال حاوی یادداشت هایی برای پخش). هر پیام کانال 
\verb;MIDI;
حاوی نوع پیام است. 
پیام، زمان و شماره یادداشت با این حال، پیام ها ممکن است حاوی اطلاعات دیگری نیز باشند
این مقاله تنها پیام های یادداشت اولیه در نظر گرفته شده است.  بخش عمده ای از پیام های 
\verb;MIDI;
عمدتاً یادداشت ها و خاموش ها هستند.  یک یادداشت نشان می دهد که یک کلید در حال فشار دادن است (یک نت در حال پخش است).  آن یادداشت ادامه خواهد داشت
برای پخش تا زمانی که پیام یادداشت آف دریافت شود،  که نشان می دهد کلیدی دیگر فشار داده نمی شود.  در داخل
پیام کانال شماره یادداشت (یعنی 
\verb;C;
= 60
وسط) و داده های دیگر مانند سرعت (حجم
توجه داشته باشید). شایان ذکر است که برخی از پیام های 
\verb;MIDI;
با نوع داده یادداشت و سرعت = 0 به عنوان یک نوشته می شوند.
جایگزینی برای نت آف، با این حال هر دو به این معنی است که پخش نت متوقف شده است. 
در 
\cref{fig.3}، اولین پیام نشان می دهد که نتی با گام = 50 و سرعت = 52 در حال پخش است.
زمان در پیام‌های یادداشتی صفر است زیرا زمان نشان می‌دهد که کلید چقدر زمان پخش شده است. را
پیام یادداشت پس از آن نشان می دهد که یادداشت 72 پس از گرفتن زمان 67.0 متوقف شده است (زمان 
\verb;midi;
سفارشی، نه
ثانیه). پس از آن، یک یادداشت روی پیام، 
\verb;note;
= 71 
را پخش می‌کند.  سپس توسط پیام یادداشت‌آف متوقف می‌شود. 
مدت زمانی که این یادداشت طول کشید 223.0 بود. 
\begin{figure}[!h]
\includegraphics[width=15cm]{3.png}
\caption{نمونه یک پیام } \label{fig.3}
\end{figure}

\subsection{شبکه‌های‌عصبی}
یک شبکه‌های‌عصبی به عنوان "مجموعه به هم پیوسته عناصر پردازش ساده (واحدها یا گره ها)" تعریف می شود.
پردازش در شبکه به وزن اتصالات بین گره هایی که توسط آنها آموزش داده می شود متکی است.
یا با مجموعه داده آموزشی تطبیق داده شود. این فرآیند همان چیزی است که اغلب آموزش شبکه یا یادگیری نامیده می شود
.  شبکه‌های‌عصبی ها ابزاری ایده آل برای مسائل طبقه بندی هستند.  شبکه‌های‌عصبی ها تطبیقی ​هستند زیرا با توجه به داده های ورودی تنظیم می شوند.  آنها قادر هستند
شناسایی توابعی که ویژگی ها را با درجه ای از دقت به دسته ها مرتبط می کنند. آن ها 
به اندازه کافی انعطاف پذیر هستند تا مشکلات پیچیده طبقه بندی دنیای واقعی مانند پیش بینی ورشکستگی، تصویر و
تشخیص گفتار و غیره را تشخیص دهند.  با این حال، همه شبکه‌های‌عصبی ها یکسان نیستند.  
شبکه‌های‌عصبی ها به دو دسته کلی تقسیم می شوند: پیشخور ، و بازگشتی.  در پیشخورها شبکه‌های‌عصبی  فعال سازی
از طریق لایه ورودی به لایه خروجی بدون هیچ گونه بازگشتی جریان می یابد.  فقط یک لایه پنهان در آن وجود دارد
بین آنها نام شبکه‌های عصبی پیشخور "عمیق" به این معنی است که لایه های پنهان بیشتری وجود دارد، اما با همان مفهوم
. ماهیت شبکه‌های‌عصبی پیشخور برای تولید موسیقی نامناسب است زیرا حافظه ندارد.
 هیچ حالت گذشته شبکه‌های‌عصبی نمی تواند روی حالت های آینده وزن ها در شبکه تأثیر بگذارد. 
نواهای های موسیقی اساسا متنی هستند و برای ایجاد نت های جدید به حافظه نت های گذشته متکی هستند.
منسجم و آهنگین (صدای دلپذیر). به همین دلیل است که شبکه‌های‌عصبی پیشخور برای تولید کننده های موسیقی کاملاً نامناسب هستند.
\section{معماری پیشنهادی}
در این مقاله، جریان نشان داده شده در
\cref{fig.121}
را پیشنهاد می کنیم.
\begin{figure}[!h]
\includegraphics[height=3cm]{4.png}
\caption{نقشه راه پیشنهادی}\label{fig.121}
\end{figure}
\subsection{تبدیل داده ها به نوع آهنگ}
تمامی فایل های 
\verb;midi;
به فرمت واسطه ای (فرمت آهنگ) که ما توسعه داده ایم، تبدیل شده اند
به ما کمک می کند تا با نت ها به روشی طبیعی تر برخورد کنیم. در حالی که روی رویدادهای هر فایل حلقه می زنیم، یادداشت ها هستند
پخش شده در تمام 16 کانال 
\verb;MIDI;
در قالب آهنگ ضبط می شود. شایان ذکر است که گاهی اوقات
فایل‌های 
\verb;MIDI; از پیام%
\verb;«note_%on»;
با «سرعت» 0 برای نشان دادن 
\verb;«note_off»; 
استفاده می‌کنند. هر رویداد 
\verb;MIDI;
 حاوی یک ویژگی زمان( 
\verb;"time";
)
است که عددی مربوط به زمان (برحسب تیک) از آخرین پیام را در خود دارد. 
فرمت آهنگ مجموعه ای از دو کلاس است که وظیفه نگهداری اطلاعات نت و تبدیل به آن را دارند. 
و از فایل های 
\verb;MIDI;
. بنابراین، قابلیت صادرات و وارد کردن
\verb;MIDI;
به فرمت کدگذاری شده انجام می شود.
داخل کلاس فرمت آهنگ( 
\verb;Song Format;)
. علاوه بر این، از آنجایی که 
\verb;MIDI;
دارای محدوده نت [0، 127] است، و ما فقط علاقه مند هستیم
در زیر مجموعه ای که کلیدهای پیانو هستند 
(زیر مجموعه [21, 21 + 88] در 
\verb;MIDI;
)، فرآیند تبدیل فرمت آهنگ
این تغییر از فضای 
\verb;MIDI;
به فضای پیانو را انجام می دهد. این فرآیند در 
\cref{fig.4}
نشان داده شده است.
\begin{figure}[!h]
\includegraphics[width=10cm,height=12cm]{5.png}
\caption{فرایند تبدیل نوع آهنگ}\label{fig.4}
\end{figure}
\subsection{رمزگذاری}
از آنجایی که هیچ راهی برای تغذیه شبکه با مقادیر 
\verb;MIDI;
به طور مستقیم و توانایی تولید موسیقی وجود ندارد،
رمزگذاری داده های موسیقی در قالبی که توسط شبکه قابل استفاده باشد بسیار مهم است. خروجی از
روش رمزگذاری مورد استفاده، ماتریسی از بردارهای باینری با اندازه 
\verb;N × 89;
برای هر فایل 
\verb;MIDI;
است که
\verb;N; 
یک عدد است. 
بردارها در قطعه 88 بیت اول در بردار با وضعیت 88 نت صفحه کلید پیانو مطابقت دارد.
صفر نشان می دهد که یک نت در این لحظه پخش نمی شود، در حالی که یک نشان دهنده عکس آن است
89  بیت در بردار تعداد تکرار است. تعداد تکرار به ما می گوید که این بردار چند بار تکرار شده است. 
یعنی چه مدت این وضعیت فعلی صفحه کلید حفظ می شود تا زمانی که تغییر کند. در آزمایشات اولیه ما،
تعداد تکرارها یک عدد طبیعی بود. با این حال، داشتن یک عدد طبیعی که نرمال نشده باشد، 
محاسبه ضرر و همگرایی را به هم‌می‌ریزد. بنابراین، در پایان از یک عدد واقعی در محدوده (0،1] استفاده کردیم. 
جایی که عدد نزدیک به 0 به معنای تکرار کمتر است و 1 به معنای بیشترین مقدار در داده 
\verb;MIDI;
است. 
فواصل نیمه باز هستند زیرا هیچ برداری نباید مقدار تکرار صفر داشته باشد، زیرا از نظر معنایی اشتباه است. 
ما آگاهانه تصمیم گرفتیم که تعداد تکرار را به جای عمومی شدن، محلی کنیم.  یعنی 1 بزرگترین باشد.
توجه داشته باشید در فایل و نه در کل مجموعه داده. استدلال ما این است که تعداد تکرار از نظر معنایی پاسخ می دهد به سوال 
«این نت نسبت به سایر نت‌های موجود در فایلش چه مدت پخش می‌شود؟» و این عدد نباید توسط بردارهای دیگر در قطعات دیگر تحت تأثیر قرار گیرد.
، حتی اگر همه این بردارها وارد فرآیند آموزش شوند. یعنی 
نت بلند در فایل نت های کوتاه نباید ارزشی کمتر از نت بلند در فایل نت های بلند داشته باشد. 
\cref{fig.5}
روش رمزگذاری مورد استفاده را نشان می دهد و نشان می دهد که بردارهای تکراری (بردارهای رنگی) چگونه به دست می آیند. 
با هم ترکیب شده عدد در تعداد تکرار (خاکستری) ثبت می شود.  تعداد تکرارها با تقسیم آن بر بزرگترین تعداد تکرار عادی می شود
.
\begin{figure}[!h]
\includegraphics[width=10cm,height=15cm]{6.png}
\caption{روش کد گذاری}\label{fig.5}
\end{figure}
\subsection{افزایش داده ها}
یک نوای خاص را می توان با هر کلید متفاوت با ندا های بسیار متفاوت نواخت اما معنایی یکسان دارد. 
معنی و به عنوان همان نوا شنیده شود. شکل 
\cref{fig.7}
در ندا موسیقی نشان می دهد که چگونه همان نوا می تواند .
با دو کلید موسیقی نواخته شود. در خط اول، نوای با کلید موسیقی "سی ماژور" پخش می شود، در حالی که در خط
دوم اینکه با کلید موسیقی "ف ماژور" پخش می شود. این دو ملودی اگر به طور جداگانه شنیده شوند، معادل هستند
اگرچه آنها کاملاً نت های متفاوتی دارند.  شکل 
\cref{fig.7}9
برای حل این مشکل از تقویت کننده روی قطعات ما استفاده شده است. 

\begin{figure}[!h]
\includegraphics[height=3cm]{7.png}
\caption{گذر از نوت سی ماژور به اف ماژور}\label{fig.7}
\end{figure}
همانطور که در \cref{fig.10} میبینیم،  پس از رمزگذاری، از هر نوا کپی هایی  می سازیم و
آنها را به طور تصادفی با انجام یک شیفت  به سمت راست جابجا می کنیم.  یعنی کلید هر همانند را طوری تغییر می دهیم که 
تعداد مشخصی همانند بدست آید.  
همانند های متمایز هر کدام در یک کلید متفاوت. دلیل استفاده از جایگشت های تصادفی این است که بتوانیم کارهای کمتری انجام دهیم. 
بیش از 11 افزایش بدون تغییر دادن کردن کلیدهای مجموعه داده. یعنی اگر سه تا افزایش بدون جهش اولیه انجام دادیم
با جابجایی، در نهایت به داده‌هایی می‌رسیم که به کلیدهای +1، +2 و +3 منتقل می‌شوند و این باعث ایجاد یک
عدم تعادل در مجموعه داده که در آن کلیدهای [+4، +11] کمتر نمایش داده می شوند. 
\begin{figure}[!h]
\includegraphics[width=7cm,height=10cm]{10.png}
\caption{نمونه یک مقاله در گوگل اسکولار}\label{fig.10}
\end{figure}
\subsection{گام های زمانی}
از آنجایی که
ل.س.ت.م
\LTRfootnote{Long-short term memory in RNN}
ها انتظار دارند که شکل ورودی در قالب (گام های زمانی، تعداد ویژگی ها) باشد، باید ما داده ها را تغییر دهیم.
داده ها باید در این فرمت باشند. برای انجام این کار از روش "پنجره کشویی" استفاده می کنیم تا در هر نمونه در
آموزش، نمونه شامل خود و
\verb;N;
نمونه قبلی است که در 
\cref{fig.9}
نشان داده شده است.
\cite{hewahi2019generation}
\begin{figure}[!h]
\includegraphics[width=7cm,height=10cm]{9.png}
\caption{نمونه یک مقاله در گوگل اسکولار}\label{fig.9}
\end{figure}
\section{خلاصه}
به طور خلاصه در 
\ref{ music}
این بخش نحوه خواندن موسیقی و تبدیل آن به زبان رایانه را دیدیم. که چگونه نوا‌ها و صداهای یم اهنگ موسیقی را به زبان رایانه تبدیل کنیم. چگونه‌مشکلات ایجاد شده در حین تبدیل کردن را بر طرف کنیم. و چگونه‌داده های به دست آورده را توسط یک شبکه عصبی تحلیل کرده و به ساخت موسیقی بپردازیم. در بخش بعدی نیز نحوه تبدیل موسیقی هایی که به زبان رایانه تبدیل شدند را به موسیقی جدید و نحوه ساخت آن را خواهیم دید.










































